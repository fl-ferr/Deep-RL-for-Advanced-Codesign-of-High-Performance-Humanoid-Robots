\chapter{Rigid Multibody Dynamics}
\label{chp:contrib_MotorDynamics}

This chapter describes the formalism and the algorithms involved in the conditioning of multibody systems behavior, taking into account motor dynamics in the framework of recursive computation methods. First, a mathematical groundwork for characterizing the dynamics of a floating-base multibody system is presented, setting a convention that is used throughout this work. Then, the problem of injecting motor parameters in the \ac{EoM} is discussed and finally the core implementation in recursive algorithms is argued.
In the forthcoming discussion, a 6D \textit{spatial vectors} notation firstly introduced by \citet{featherstone_rigid_2008} is presented. This will be used to describe the kinematics and dynamics of a floating-base multibody system in a unified manner.

\paragraph{Spatial Vectors} A spatial vector is a 6D vector that describes the motion of a rigid body in space.

In the case of a rigid body, the velocity of a point $P$ attached to the body respect to a reference frame attached to an arbitrary point $O$ in the space can be generally expressed by its angular component $\mathbf{\omega}$ about an axis passing through $O$ and its linear component $\mathbf{v} _P$, for which the following relation holds:

\begin{equation}
    v _P = \mathbf{\omega} \times \bar{OP}
\end{equation}

where $\bar{OP}$ is the position vector of $P$ with respect to $O$. This holds for any point $P$ on the rigid body. In order to simplify the notation, introducing a Cartesian coordinate frame $\mathcal{O} _{xyz}$, we can define a basis of 6 spatial vectors $\mathcal{D} _O = \{\mathbf{d} _i\} ^6 _{i=1}$ as:

\begin{equation}
    \mathcal{D} _O = \{ \mathbf{d} _{O _x}, \mathbf{d} _{O _y}, \mathbf{d} _{O _z}, \mathbf{d} _x, \mathbf{d} _y, \mathbf{d} _z \} \subset \mathcal{M} ^6
\end{equation}

where $\mathcal{M} ^6$ is the space of 6D vectors, defining a Pl\"ucker coordinate system on $\mathcal{M} ^6$.


\dots

\paragraph{Spatial Velocity} \dots The spatial velocity of a rigid body is defined as:

\begin{equation}
    \mathbf{v} _P = \begin{bmatrix}
        \mathbf{v} _P \\
        \boldsymbol{\omega} _P
    \end{bmatrix}
\end{equation}

where $\mathbf{v} _P$ is the linear velocity of the point $P$ and $\boldsymbol{\omega} _P$ is the angular velocity of the body.

\paragraph{Spatial Forces} \dots The spatial force acting on a rigid body is defined as:

\begin{equation}
    \mathbf{f} = \begin{bmatrix}
        \mathbf{f} \\
        \boldsymbol{\tau}
    \end{bmatrix}
\end{equation}



\section{Problem Formalization}

Starting from the equation of motion of a robot manipulator:

\begin{equation}
    \mathbf{M}(q)\dot{\boldsymbol{\nu}} + \mathbf{h}(q,\boldsymbol{\nu}) = \mathbf{B}\boldsymbol{\tau} + \mathbf{J} ^T \mathbf{f}
\end{equation}

where:

\begin{itemize}
    \item $\mathbf{M}(q)$ is the inertia matrix
    \item $\mathbf{h}(q,\boldsymbol{\nu})$ is the Coriolis vector
    \item $\mathbf{B}$ is the actuation matrix
    \item $\boldsymbol{\tau}$ is actuation torques vector
    \item $\mathbf{J}$ is the Jacobian matrix
    \item $\mathbf{f}$ is the external forces vector
\end{itemize}

we can isolate the terms related to the base link (usually in position 0) from the joints' poses:

\begin{align}
    \boldsymbol{\nu} =
    \begin{bmatrix}
        \mathrm{\mathbf{v}} \\
        \dot{\mathbf{s}}
    \end{bmatrix} &  &
    \dot{\boldsymbol{\nu}} =
    \begin{bmatrix}
        \dot{\mathrm{\mathbf{v}}} \\
        \ddot{\mathbf{s}}
    \end{bmatrix}
\end{align}

where $\mathrm{\mathbf{v}} \in \mathbb{R} ^{6}$ and $\mathbf{s} \in \mathbb{R}^{N_B}$, we get to the form:

\begin{equation}
    \begin{bmatrix}
        \mathbf{M} _{\mathcal{B}}(q)     & \mathbf{M} _{\mathcal{B}S}(q) \\
        \mathbf{M} _{\mathcal{B}S} ^T(q) & \mathbf{M} _s(q)
    \end{bmatrix}
    \begin{bmatrix}
        \dot{\mathrm{\mathbf{v}}} \\
        \ddot{\mathbf{s}}
    \end{bmatrix}+
    \begin{bmatrix}
        \mathbf{h} _{\mathcal{B}} \\
        \mathbf{h} _S
    \end{bmatrix}=
    \begin{bmatrix}
        \mathbb{0} \\
        \mathbb{1}
    \end{bmatrix}
    \boldsymbol{\tau}
    +
    \begin{bmatrix}
        \mathbf{J} _{\mathcal{B}} \\
        \mathbf{J} _S
    \end{bmatrix} ^T
    \mathbf{f}
\end{equation}

Given that the dynamics of the set of motors can be described by the following equation:

\begin{equation}
    \label{eqn:mot_dyn}
    \mathbf{I} _R \ddot{\boldsymbol{\theta}} + \mathbf{K}_v \dot{\boldsymbol{\theta}} = \boldsymbol{\tau}_m
\end{equation}

where $\mathbf{K _v}$ is the diagonal matrix of motor viscous coefficients and $\mathbf{I}_R$ is the diagonal matrix of motors' inertias. Considering that given the set of transmission ratios $\boldsymbol{\Gamma}$, the relation between the joints' velocities and the motors' velocities is:

\begin{align}
    \mathbf{s} = \boldsymbol{\theta} \boldsymbol{\Gamma} &  & \dot{\mathbf{s}} = \dot{\boldsymbol{\theta}} \boldsymbol{\Gamma} &  & \ddot{\mathbf{s}} = \ddot{\boldsymbol{\theta}} \boldsymbol{\Gamma}
\end{align}

we can rewrite the equation \ref{eqn:mot_dyn} in the joints' space as:

\begin{equation}
    \label{eqn:mot_dyn_jointspace}
    \boldsymbol{\tau} = \boldsymbol{\Gamma} ^{-T} (\mathbf{I} _R\boldsymbol{\Gamma} ^{-1} \ddot{s} + \mathbf{K}_v \boldsymbol{\Gamma} ^{-1}\dot{s})
\end{equation}

Therefore, the \ac{EoM} of the multibody system can be rewritten as:

\begin{equation}
    \underbrace{\begin{bmatrix}
            \mathbf{M} _{\mathcal{B}}(q)     & \mathbf{M} _{\mathcal{B}S}(q)                                                      \\
            \mathbf{M} _{\mathcal{B}S} ^T(q) & \mathbf{M} _s(q) + \boldsymbol{\Gamma} ^{-T}\mathbf{I} _R\boldsymbol{\Gamma} ^{-1}
        \end{bmatrix}} _{\mathbf{\bar{M}}(q)}
    \begin{bmatrix}
        \dot{\mathrm{\mathbf{v}}} \\
        \ddot{\mathbf{s}}
    \end{bmatrix}+
    \mathbf{h}
    (q,\boldsymbol{\nu}) =
    \underbrace{\begin{bmatrix}
            \mathbb{0} \\
            \boldsymbol{\Gamma} ^{-T}
        \end{bmatrix}} _{\mathbf{\bar{B}}}
    \boldsymbol{\tau} _m
    +
    \mathbf{J} ^T
    \mathbf{f}
    -
    \underbrace{\begin{bmatrix}
            \mathbb{0} \\
            \boldsymbol{\Gamma} ^{-T}\mathbf{K _v}\boldsymbol{\Gamma} ^{-1}
        \end{bmatrix}} _\mathbf{\bar{K _v}}
    \begin{bmatrix}
        \mathrm{\mathbf{v}} \\
        \dot{\mathbf{s}}
    \end{bmatrix}
\end{equation}

or, in a more compact form that will be used for computation as:

\begin{equation}
    \mathbf{\bar{M}}(q)\dot{\boldsymbol{\nu}} + \mathbf{h}(q,\boldsymbol{\nu}) = \mathbf{\bar{B}}\boldsymbol{\tau} _m + \mathbf{J} ^T \mathbf{f} - \bar{\mathbf{K _v}}\boldsymbol{\nu}
\end{equation}

