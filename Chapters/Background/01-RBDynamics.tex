\chapter{Rigid Multibody Dynamics}
\label{chp:back_RBDynamics}

In this chapter, a mathematical groundwork for characterizing the dynamics of a floating-base multibody system is presented, setting a convention that is used throughout this work. The notation will be used to describe the kinematics and dynamics of a set of rigid bodies linked between each other, putting the basis for a further discussion on the derivation of computationally efficient algorithms for the computation of the dynamics. Then, a first introduction to recursive rigid body algorithms (\ac{RBDA}) for of a multibody system is presented, to give a comprehensive understanding of the state of the art in the field of rigid multibody dynamics algorithms. In the forthcoming discussion, we ignore the 6D \textit{spatial vectors} notation firstly introduced by \citet{featherstone_rigid_2008} for the original formulation of the \textit{Articulated-Body Algorithm}, preferring instead a more rich notation that includes information about the expressing frames used, which can be less confusing and less-likely error-prone when dealing with complex recursive algorithms.

\section{Notation and Conventions}

\begin{itemize}
    \item The set $\mathrm{SE}(3)$ is the \textit{special Euclidean} Lie group defined as:

          \begin{equation}
              \mathrm{SE}(3) := \left\{
              \begin{bmatrix}
                  \mathbf{R}                & \mathbf{p} \\
                  \mathbbm{0} _{1 \times 3} & 1
              \end{bmatrix} \in \mathbb{R} ^4 \mid \mathbf{R} \in \mathrm{SO}(3), \mathbf{p} \in \mathbb{R} ^3
              \right\}
          \end{equation}

    \item The set $\mathrm{SO}(3)$ is the \textit{special orthogonal} Lie group defined as:

          \begin{equation}
              \label{eqn:SO3}
              \mathrm{SO}(3) := \left\{
              \mathbf{R} \in \mathbb{R} ^{3 \times 3} \mid \mathbf{R} ^\top \mathbf{R} = \mathbb{I}, \det(\mathbf{R}) = 1
              \right\}
          \end{equation}

    \item A smooth manifold $\mathbb{Q}$ is a set of elements $\mathbb{Q} = \{ \mathbf{q} \}$ that locally resembles an Euclidean space $\mathbb{R} ^n$.

    \item The derivative of an element of a smooth manifold $\mathbb{Q} : \mathrm{SE}(3)$ is an element of its tangent space $\mathbb{T} _{\mathbb{Q}}$, i.e.:

          \begin{equation}
              \mathbb{T} _{\mathbb{Q}} := \left\{
              \begin{bmatrix}
                  \mathbf{S}                & \mathbf{v} \\
                  \mathbbm{0} _{1 \times 3} & 0
              \end{bmatrix} \in \mathbb{R} ^6 \mid \mathbf{S} \in \mathfrak{so}(3), \mathbf{v} \in \mathbb{R} ^3
              \right\}
          \end{equation}

    \item The set $\mathfrak{so}(3)$ is the \textit{special orthogonal} Lie algebra of $\mathrm{SO}(3)$, defined as the set of $3\times3$ \textit{skew-symmetric} matrices:

          \begin{equation}
              \mathfrak{so}(3) = \{S \in \mathbb{R}^{3\times3} \mid S^\top = -S \}
          \end{equation}

    \item The skew-symmetric matrix of a vector $\mathbf{v} = (v_1, v_2, v_3) \in \mathbb{R}^3$ as:

          \begin{equation}
              \mathbf{v} ^\wedge = \begin{bmatrix}
                  0    & -v_3 & v_2  \\
                  v_3  & 0    & -v_1 \\
                  -v_2 & v_1  & 0
              \end{bmatrix}
          \end{equation}

    \item Similarly, for a $6$D vector $\mathbf{v} = (v, \omega) \in \mathbb{R}^6$, where $v \in \mathbb{R}^3$ and $\omega \in \mathbb{R}^3$, we can define its skew-symmetric matrix $\mathbf{v} ^\wedge \in \mathfrak{se}(3)$ as:

          \begin{equation}
              \mathbf{v} ^\wedge = \begin{bmatrix}
                  \omega ^\wedge            & v \\
                  \mathbbm{0} _{1 \times 3} & 0
              \end{bmatrix}
          \end{equation}
\end{itemize}

\section{Robot Modelling}

\subsection{Frames, Rotation and Transformation Matrices}

Considering two frames sharing the same origin $A, B \in \mathbb{R}^3$ and assuming their bases to be orthonormal. As a consequence, once defined ${}^F\mathbf{a} _i \in \mathbb{R}^3$ as the $i$-th base of the frame $A$ expressed in arbitrary frame $F$ and equivalently ${}^F\mathbf{a} _i \in \mathbb{R}^3$ as the $i$-th base of the frame $B$ expressed in the same frame $F$. In the same fashion, given a point $\mathbf{p}$ whose coordinates are defined in frame $B$, namely ${}^B\mathbf{p}$ we obtain the corresponding coordinates in frame $A$ as follows:

\begin{equation}
    {}^A\mathbf{p} = \begin{bmatrix} {}^Ab_1 & {}^Ab_2 & {}^Ab_3 \end{bmatrix} {}^B \mathbf{p} = {}^A \mathbf{R}_B {}^B\mathbf{p}
\end{equation}

in which ${}^A\mathbf{R}_B$ is a rotation matrix belonging to $\mathrm{SO}(3)$ as per \cref{eqn:SO3}. Let us now consider an external frame $F$ that \textit{does not} share the origin with $A$ and $B$, by defining ${}^A\mathbf{o}_F$ the position of the frame $F$ in the origin measured in frame $A$ and by defining the transform of coordinates of a point ${}^F\mathbf{q}$ into $A$ as follows:

\begin{equation}
    {}^A \mathbf{q} = {}^A \mathbf{R}_F {}^F \mathbf{q} + {}^A \mathbf{o}_F
\end{equation}

we can get to a more compact form by defining the \textit{homogeneous transformation} matrix ${}^A\mathbf{H}_F \in \mathrm{SE}(3)$ as:

\begin{equation}
    \begin{bmatrix}
        {}^A \mathbf{q} \\ 1
    \end{bmatrix} =
    \begin{bmatrix}
        {}^A \mathbf{R}_F & {}^A \mathbf{o}_F \\ 0 & 1
    \end{bmatrix} \qquad \rightarrow \qquad {}^A\mathbf{q} = {}^A \mathbf{H}_F {}^F \mathbf{q}
\end{equation}

Thus in the hypothesis of frame rigidly attached to a body, the homogeneous transformation can be used to describe their relative pose.

\subsection{Rigid Body Velocity}

Let us now define $A$ and $B$ as two frames in a relative motion with respect to each other. The temporal derivative of their \textit{relative transformation} can be expressed as:

\begin{equation}
    \label{eqn:transformation_derivative}
    {}^A\dot{\mathbf{H}}_B := \frac{d}{dt} \left( {}^A\mathbf{H} _B \right) =
    \begin{bmatrix}
        {}^A\dot{\mathbf{R}}_B    & {}^A\dot{\mathbf{o}}_B \\
        \mathbbm{0} _{1 \times 3} & 0
    \end{bmatrix}
\end{equation}

A more compact notation can be obtained by multiplying it on the left by the inverse of the transformation matrix ${}^A\mathbf{H} _B$:

\begin{equation}
    {}^A\dot{\mathbf{H}} _B ^{-1} {}^A\dot{\mathbf{H}} _B =
    \begin{bmatrix}
        {}^A\mathbf{R}_B^\top     & -{}^A\mathbf{R}_B^\top {}^A\mathbf{o}_B \\
        \mathbbm{0} _{1 \times 3} & 1
    \end{bmatrix}
    \begin{bmatrix}
        {}^A\dot{\mathbf{R}}_B    & {}^A\dot{\mathbf{o}}_B \\
        \mathbbm{0} _{1 \times 3} & 0
    \end{bmatrix} =
    \begin{bmatrix}
        {}^A\mathbf{R}_B^\top {}^A\dot{\mathbf{R}}_B & {}^A\mathbf{R}_B^\top {}^A\dot{\mathbf{o}}_B \\
        \mathbbm{0} _{1 \times 3}                    & 0
    \end{bmatrix}
\end{equation}

in which the product ${}^A\mathbf{R}_B^\top {}^A\dot{\mathbf{R}}_B$ is an element of $\mathrm{SO}(3)$. The \cref{eqn:transformation_derivative} can be described by two three-dimensional velocity vectors representing the linear and the angular components:

\begin{align}
    {}^A\mathbf{v}_{A,B} = {}^A\mathbf{R}_B^\top = {}^A\mathbf{R}_B^\top {}^A\dot{\mathbf{o}}_B \\
    {}^A\omega_{A,B} = {}^A\mathbf{R}_B^\top {}^A\mathbf{R}_B = {}^A\mathbf{R}_B^\top {}^A\dot{\mathbf{R}}_B
\end{align}

and therefore we can serialize the two components obtaining the \textit{left trivialized} $6$D velocity vector in \textit{body representation}:

\begin{equation}
    \label{eqn:left_trivialized}
    {}^A\mathbf{v}_{A,B} = \begin{bmatrix}
        {}^A\mathbf{v}_{A,B} \\
        {}^A\omega_{A,B}
    \end{bmatrix}
\end{equation}

Similarly, by defining the $6$D velocity vector such that the linear component is equal to the time derivative of the relative position of the two frames, we obtain the \textit{mixed representation} of the velocity \citep{traversaro_modelling_2019}:

\begin{equation}
    {}^{B[A]}\mathbf{v} _{C,D} =
    \begin{bmatrix}
        {}^B \dot{\mathbf{o}} _B \\
        {}^B \dot{\omega}_{C,D}
    \end{bmatrix} =
    {}^{C[A]}_B {}^B\mathbf{v}_{A,B} = \begin{bmatrix}
        {}^{C[A]}_B {}^B\mathbf{v}_{A,B} \\
        {}^{C[A]}_B {}^B\omega_{A,B}
    \end{bmatrix}
\end{equation}

\subsection{Accelerations and Forces}

The acceleration between two frames $A$ and $B$ in relative motion can be expressed as the time derivative of their relative velocity:

\begin{equation}
    {}^C\dot{\mathbf{v}} = \frac{d}{dt}\left( {}^C \mathbf{v}_{A,B}\right)
\end{equation}

By recalling the definition for the left trivialized velocity defined in \cref{eqn:left_trivialized}, we obtain:

\begin{equation}
    {}^C\dot{\mathbf{v}} _{A,B} = \frac{d}{dt} \left({}^C \mathbf{X}_B {}^B\mathbf{v}_{A,B} \right) = {}^C \mathbf{X}_B \dot{\mathbf{v}}_{A,B} + {}^C \dot{\mathbf{X}}_B {}^B\mathbf{v}_{A,B}
\end{equation}

In case in which the measurement frame $C$ corresponds to either $A$ or $B$, the time derivative of the velocity transform $\dot{\mathbf{X}}_B$ is equal to zero, thus obtaining the so called \textit{intrinsic acceleration}:

\begin{equation}
    {}^C\mathbf{a}_{A,B} = {}^C \mathbf{X}_B \dot{\mathbf{v}}_{A,B}
\end{equation}

The force acting on a rigid body is defined as a stacking of the linear component $f \in \mathbb{R}^3$ and the angular component $\tau \in \mathbb{R}^3$ of the wrench acting on the body, in particular, we can defined $\mathbf{f} \in \mathbb{R}^6$ respect to a frame $B$ as:

\begin{equation}
    {}_{B}\mathbf{f} = \begin{bmatrix}
        {}_{B}f \\
        {}_{B}\tau
    \end{bmatrix}
\end{equation}

In particular, a change of reference frame can be expressed as:

\begin{equation}
    {}_{A}\mathbf{f} = {}_{A}\mathbf{X}^B {}_{B}\mathbf{f}
\end{equation}

in which ${}_{A}\mathbf{X}_B = {}^A\mathbf{X}_B ^{-\top}$ is the $6$D forces transformation matrix refined as:

\begin{equation}
    \begin{bmatrix}
        {}^A\mathbf{R}_B                         & \mathbbm{0}_{3 \times 3} \\
        {}^A\mathbf{o}_B^\wedge {}^A\mathbf{R}_B & {}^A\mathbf{R}_B
    \end{bmatrix}
\end{equation}

\section{Floating Base Multibody System Modelling}

\subsection{Joints and Links}

A multibody kinematic tree can be described as a contiguous assembly of two main physical elements: links and joints. The links compose the inertial properties of the system, while the joints are usually modeled as massless elements connecting two links, allowing relative motion between them. They are classified according to the number of \ac{DoF} they allow, which is the number of independent coordinates needed to describe the relative motion between the two links, usually assuming values between 0 (fixed joint) and 6 (free joint).

For the sake of simplicity and without loss of generality, in this work, only 1-\ac{DoF} joint will be considered, as their properties can be extended to multidimensional cases. Furthermore, the system will be considered \textit{acyclic}, i.e. it will not contain any closed loop, as the presence of loops would make the system kinematically redundant.
Given a joint we can define a \textit{joint axis} $\mathbf{a} \in \mathbb{R}^3, |\mathbf{a}| = 1$, in this case, if we recall the definition of the spatial subspace, we can define a joint motion subspace base $\boldsymbol{\Phi} _j \in \mathbb{R} ^6$ as:

\begin{equation}
    \label{eqn:subspace}
    \boldsymbol{\Phi} _j =
    \begin{cases}
        [\mathbbm{0}_{1 \times 6}] ^\top             & \text{if the joint is fixed}     \\
        [\mathbf{a}, \mathbbm{0}_{1 \times 3}] ^\top & \text{if the joint is prismatic} \\
        [\mathbbm{0}_{1 \times 3}, \mathbf{a}] ^\top & \text{if the joint is revolute}
    \end{cases}
\end{equation}


\subsection{Poses and Velocities}

When dealing with multibody systems, the configuration is described by a configuration belonging to a smooth manifold $\mathbf{H} \in \mathbb{Q} : \mathrm{SE}(3) \times \mathbb{R}^n$ where $n$ is the number of degrees of freedom of the system. The velocity of the system is described by the time derivative of the pose $\mathbf{H}$, which belongs to the tangent space of the manifold $\mathbb{Q}$, i.e. $\dot{\mathbf{H}} \in \mathbb{T}_{\mathbb{Q}}$. In the case of a floating-base multibody system, we should take into account the presence of joints, therefore we can reformulate the generalized velocity by stacking the joint velocities $\dot{\mathbf{s}} \in \mathbb{R}^n$ and the velocity of the base respect to an inertial frame $\mathcal{I}$, i.e. ${}^C\mathbf{v}_{\mathcal{I},\mathcal{B}} \in \mathbb{R}^6$, as:

\begin{equation}
    {}^C \boldsymbol{\nu} =
    \begin{bmatrix}
        {}^C \mathbf{v}_{\mathcal{I},\mathcal{B}} \\
        \dot{\mathbf{s}}
    \end{bmatrix} \in \mathbb{R}^{6+n}
\end{equation}

in which ${}^C \mathbf{v}_{\mathcal{I},B}$ is the velocity of the base link, $\dot{\mathbf{s}}$ is the vector of joint velocities, and $C$ is the generic frame in which the velocity is expressed.


\section{Equation of Motion}
\label{sec:back_eom}

The equation of motion of a generic multibody mechanical system can be derived starting from the Principle of Least Action, which states that in a time interval $[t _0, t _1]$ the motion of a system is such that the action functional $\mathcal{A}$ is stationary, i.e. the first variation of the action functional is zero:

\begin{equation}
    \mathcal{A} = \int _{t _0} ^{t _1} \mathcal{L} (\mathbf{q}(t), \mathbf{\dot{q}}(t)) dt \quad \text{with } \delta \mathcal{A} = 0
\end{equation}

where $\mathcal{L} (\mathbf{q} = {}^W\mathbf{H} _B, \dot{\mathbf{q}} = {}^W\dot{\mathbf{H}} _B)$ is the Lagrangian of the system. For a floating-base multibody system, it can be obtained as the sum of the Lagrangian of each link composing the system:

\begin{equation}
    \mathcal{L} = \mathcal{T}(\mathbf{q},\boldsymbol{\nu}) - \mathcal{U}(\mathbf{q}) = \sum _{L \in \mathbb{L}} \mathrm{v} ^\top _L \mathbb{M} _L \mathrm{v} _L - \sum _{L \in \mathbb{L}} \begin{bmatrix}
        \mathbf{g} ^\top _L & \mathbbm{0}
    \end{bmatrix} {}^W\mathbf{H} _L
    \begin{bmatrix}
        m _L c _L \\ m _L
    \end{bmatrix}
\end{equation}

where $\mathbb{L}$ is the set of links composing the system, $\mathbb{M} _L \in \mathbb{R} ^{6 \times 6}$ is the spatial inertia matrix of the link $L$, $\mathbf{g} _L$ is the gravitational acceleration vector, ${}^W\mathbf{H} _L \in \mathbb{Q}$ is the pose of the link $L$ and $m _L \in \mathbb{R}$ and $c _L \in \mathbb{R}$ are the mass and the displacement between the body's \ac{CoM} and the origin of the link $L$, respectively.

The first variation of the action functional can be written as:

\begin{equation}
    \delta \mathcal{A} = \int _{t _0} ^{t _1} \delta \mathcal{L} (\mathbf{q}(t), \mathbf{\dot{q}}(t))dt = \int _{t _0} ^{t _1} \left( \frac{\partial \mathcal{L}}{\partial \mathbf{q}} \delta \mathbf{q} + \frac{\partial \mathcal{L}}{\partial \mathbf{\dot{q}}} \delta \mathbf{\dot{q}} \right) dt
\end{equation}

where $\delta \mathbf{q}$ and $\delta \mathbf{\dot{q}}$ are the variations of the generalized coordinates and velocities, respectively. The first variation of the action functional is zero if and only if the integrand is zero. In the case of Euclidean spaces, i.e. $\mathbb{Q} = \mathbb{T} _{\mathbb{Q}} =\mathbb{R} ^n$, this yields the Euler-Lagrange equation:

\begin{equation}
    \frac{d}{dt} \frac{\partial \mathcal{L}}{\partial \mathbf{\dot{q}}} - \frac{\partial \mathcal{L}}{\partial \mathbf{q}} = 0
    \label{eqn:lagrangian}
\end{equation}

The following chapters will always consider a rigid body pose and body-fixed velocities, thus yielding the \textit{left trivialized Lagrangian} \citep{traversaro_modelling_2019}:

\begin{equation}
    l(\mathbf{H}, \mathbf{v}) = \mathcal{L}(\mathbf{H}, \mathbf{H}\mathbf{v}^\wedge) = \mathcal{T}(\mathbf{v}) - \mathcal{U}(\mathbf{H})
\end{equation}

Once defined a map $\mathbb{M} \in \mathbb{R}^{(6+n) \times (6+n)}$ as the generalized inertia matrix for the kinetic energy $\mathcal{T}$, a bias forces vector $\mathbf{h} \in \mathbb{R} ^{6+n}$ including the contribution of Coriolis and gravitational forces, a selection matrix $\mathbf{B} = \begin{bmatrix} \mathbbm{0}_{n\times6} & \mathbbm{1}_n \end{bmatrix}^\top$ for the actuation torques $\boldsymbol{\tau} \in \mathbb{R} ^n$, the stacked flaoting-base Jacobian matrix $\mathbf{J} _\text{c} \in \mathbb{R} ^{(6+n) \times (6+n_c)}$ for the external forces $\mathbf{F} ^\text{ext} \in \mathbb{R} ^{6n_c}$ with $n_c$ number of contact frames, the equation of motion can be finally obtained \citep{siciliano_handbook_2008}:

\begin{equation}
    \label{eqn:equation_of_motion}
    \mathbb{M}(\mathbf{q}) \mathbf{\dot{\boldsymbol{\nu}}} + \mathbf{h} (\mathbf{q}, \boldsymbol{\nu}) = \mathbf{B}\boldsymbol{\tau} + \mathbf{J}^\top _\text{c} \mathbf{F} ^\text{ext}
\end{equation}


\section{Forward Dynamics}
\label{sec:back_fd}

The forward dynamics problem consists of computing the joint accelerations of a multibody system given the joint positions, velocities, and applied forces. The computation of the acceleration of the links normally involves the inversion of the mass matrix, which can easily become huge in a multibody system with a high number of links, in fact, the mass matrix of a system with $n$ links is a $6+n \times 6+n$ matrix, as explained in \cref{sec:back_eom}. Although being algorithms involving the inertia matrix inversion much simpler to implement, propagation algorithms involving recursive backsubstition are more efficient in terms of computational cost, as they can reach the theoretical minimal complexity of $\mathcal{O}(n)$. In this work, the notation $\mathrm{FD}(\cdot)$ will be used to indicate the forward dynamics computation.

\begin{equation}
    \ddot{\mathbf{s}} = \mathrm{FD} (\mathcal{M}, \mathbf{s}, \dot{\mathbf{s}}, \boldsymbol{\tau}, \mathbf{F} ^{\text{ext}})
\end{equation}

\subsection{Articulated-Body Algorithm}
\label{subsec:back_aba}

As anticipated, the inversion of the mass matrix for a multi-body system with a high number of links is computationally expensive, in fact, even using \citet{coppersmith_matrix_1990} algorithm or its optimized version by \citet{vassilevska-williams2012breaking}, which reached a complexity of $\mathcal{O}(n^{2.373})$, the computation of the inverse of the mass matrix is still too expensive for real-time scenarios. That is why recursive methods results are more suitable for applications where a fast computation of the dynamics is required. In recursive algorithms, as the problem is unsolvable for a single body of the system, the usual approach is to define the constraints that the links must satisfy, write the equation for each link, calculate the local coefficient, and then propagate them in the kinematic tree until a point in which the problem becomes solvable, usually the base of the tree in backpropagation or the end-effector in forward propagation.

% === FIG: Kinematic Chain === %
\begin{figure}
    \centering
    \caption{Kinematic subtree visualization.}
    \label{fig:kin_tree}
    \subfloat[Branched kinematic tree]{
        \resizebox{0.45\textwidth}{!}{
            \tikzset {_wvafpciex/.code = {\pgfsetadditionalshadetransform{ \pgftransformshift{\pgfpoint{0 bp } { 0 bp }  }  \pgftransformrotate{-117 }  \pgftransformscale{2 }  }}}
\pgfdeclarehorizontalshading{_sy15ycl19}{150bp}{rgb(0bp)=(1,1,1);
    rgb(37.5bp)=(1,1,1);
    rgb(50.08184160505022bp)=(0.95,0.95,0.95);
    rgb(57.64583042689732bp)=(0.88,0.88,0.88);
    rgb(61.33184160505022bp)=(0.96,0.96,0.96);
    rgb(100bp)=(0.96,0.96,0.96)}
\tikzset {_svf4mjzjm/.code = {\pgfsetadditionalshadetransform{ \pgftransformshift{\pgfpoint{0 bp } { 0 bp }  }  \pgftransformrotate{-117 }  \pgftransformscale{2 }  }}}
\pgfdeclarehorizontalshading{_e9qihdtow}{150bp}{rgb(0bp)=(1,1,1);
    rgb(37.5bp)=(1,1,1);
    rgb(50.08184160505022bp)=(0.95,0.95,0.95);
    rgb(57.64583042689732bp)=(0.88,0.88,0.88);
    rgb(61.33184160505022bp)=(0.96,0.96,0.96);
    rgb(100bp)=(0.96,0.96,0.96)}
\tikzset {_ddhli8fcf/.code = {\pgfsetadditionalshadetransform{ \pgftransformshift{\pgfpoint{0 bp } { 0 bp }  }  \pgftransformrotate{-117 }  \pgftransformscale{2 }  }}}
\pgfdeclarehorizontalshading{_bw97u0oo5}{150bp}{rgb(0bp)=(1,1,1);
    rgb(37.5bp)=(1,1,1);
    rgb(50.08184160505022bp)=(0.95,0.95,0.95);
    rgb(57.64583042689732bp)=(0.88,0.88,0.88);
    rgb(61.33184160505022bp)=(0.96,0.96,0.96);
    rgb(100bp)=(0.96,0.96,0.96)}
\tikzset {_rzri9nu6o/.code = {\pgfsetadditionalshadetransform{ \pgftransformshift{\pgfpoint{0 bp } { 0 bp }  }  \pgftransformrotate{-117 }  \pgftransformscale{2 }  }}}
\pgfdeclarehorizontalshading{_irmxrz6bl}{150bp}{rgb(0bp)=(1,1,1);
    rgb(37.5bp)=(1,1,1);
    rgb(50.08184160505022bp)=(0.95,0.95,0.95);
    rgb(57.64583042689732bp)=(0.88,0.88,0.88);
    rgb(61.33184160505022bp)=(0.96,0.96,0.96);
    rgb(100bp)=(0.96,0.96,0.96)}
\tikzset {_tv3cr0nbo/.code = {\pgfsetadditionalshadetransform{ \pgftransformshift{\pgfpoint{0 bp } { 0 bp }  }  \pgftransformrotate{-117 }  \pgftransformscale{2 }  }}}
\pgfdeclarehorizontalshading{_ria03yqfs}{150bp}{rgb(0bp)=(1,1,1);
    rgb(37.5bp)=(1,1,1);
    rgb(50.08184160505022bp)=(0.95,0.95,0.95);
    rgb(57.64583042689732bp)=(0.88,0.88,0.88);
    rgb(61.33184160505022bp)=(0.96,0.96,0.96);
    rgb(100bp)=(0.96,0.96,0.96)}
\tikzset {_0hgwncxm6/.code = {\pgfsetadditionalshadetransform{ \pgftransformshift{\pgfpoint{0 bp } { 0 bp }  }  \pgftransformrotate{-117 }  \pgftransformscale{2 }  }}}
\pgfdeclarehorizontalshading{_7fixrohbn}{150bp}{rgb(0bp)=(1,1,1);
    rgb(37.5bp)=(1,1,1);
    rgb(50.08184160505022bp)=(0.95,0.95,0.95);
    rgb(57.64583042689732bp)=(0.88,0.88,0.88);
    rgb(61.33184160505022bp)=(0.96,0.96,0.96);
    rgb(100bp)=(0.96,0.96,0.96)}
\tikzset {_te42shfqc/.code = {\pgfsetadditionalshadetransform{ \pgftransformshift{\pgfpoint{0 bp } { 0 bp }  }  \pgftransformrotate{-117 }  \pgftransformscale{2 }  }}}
\pgfdeclarehorizontalshading{_9prsrmsl7}{150bp}{rgb(0bp)=(1,1,1);
    rgb(37.5bp)=(1,1,1);
    rgb(50.08184160505022bp)=(0.95,0.95,0.95);
    rgb(57.64583042689732bp)=(0.88,0.88,0.88);
    rgb(61.33184160505022bp)=(0.96,0.96,0.96);
    rgb(100bp)=(0.96,0.96,0.96)}
\tikzset {_qoggn7fn2/.code = {\pgfsetadditionalshadetransform{ \pgftransformshift{\pgfpoint{0 bp } { 0 bp }  }  \pgftransformrotate{-117 }  \pgftransformscale{2 }  }}}
\pgfdeclarehorizontalshading{_8g14z78sd}{150bp}{rgb(0bp)=(1,1,1);
    rgb(37.5bp)=(1,1,1);
    rgb(50.08184160505022bp)=(0.95,0.95,0.95);
    rgb(57.64583042689732bp)=(0.88,0.88,0.88);
    rgb(61.33184160505022bp)=(0.96,0.96,0.96);
    rgb(100bp)=(0.96,0.96,0.96)}
\tikzset {_3ytgx3jcr/.code = {\pgfsetadditionalshadetransform{ \pgftransformshift{\pgfpoint{0 bp } { 0 bp }  }  \pgftransformrotate{-117 }  \pgftransformscale{2 }  }}}
\pgfdeclarehorizontalshading{_hjawm1m3k}{150bp}{rgb(0bp)=(1,1,1);
    rgb(37.5bp)=(1,1,1);
    rgb(50.08184160505022bp)=(0.95,0.95,0.95);
    rgb(57.64583042689732bp)=(0.88,0.88,0.88);
    rgb(61.33184160505022bp)=(0.96,0.96,0.96);
    rgb(100bp)=(0.96,0.96,0.96)}
\tikzset {_Am47npqyx/.code = {\pgfsetadditionalshadetransform{ \pgftransformshift{\pgfpoint{0 bp } { 0 bp }  }  \pgftransformrotate{-117 }  \pgftransformscale{2 }  }}}
\pgfdeclarehorizontalshading{_hbtcw454n}{150bp}{rgb(0bp)=(1,1,1);
    rgb(37.5bp)=(1,1,1);
    rgb(50.08184160505022bp)=(0.95,0.95,0.95);
    rgb(57.64583042689732bp)=(0.88,0.88,0.88);
    rgb(61.33184160505022bp)=(0.96,0.96,0.96);
    rgb(100bp)=(0.96,0.96,0.96)}
\tikzset {_igkxxgbt5/.code = {\pgfsetadditionalshadetransform{ \pgftransformshift{\pgfpoint{0 bp } { 0 bp }  }  \pgftransformrotate{-117 }  \pgftransformscale{2 }  }}}
\pgfdeclarehorizontalshading{_4ks716t6n}{150bp}{rgb(0bp)=(1,1,1);
    rgb(37.5bp)=(1,1,1);
    rgb(50.08184160505022bp)=(0.95,0.95,0.95);
    rgb(57.64583042689732bp)=(0.88,0.88,0.88);
    rgb(61.33184160505022bp)=(0.96,0.96,0.96);
    rgb(100bp)=(0.96,0.96,0.96)}
\tikzset{every picture/.style={line width=0.75pt}}
\begin{tikzpicture}[x=0.75pt,y=0.75pt,yscale=-1,xscale=1]
    \path  [shading=_sy15ycl19,_wvafpciex] (340.92,97.33) .. controls (340.92,88.96) and (364.78,82.17) .. (394.21,82.17) .. controls (423.64,82.17) and (447.5,88.96) .. (447.5,97.33) .. controls (447.5,105.71) and (423.64,112.5) .. (394.21,112.5) .. controls (364.78,112.5) and (340.92,105.71) .. (340.92,97.33) -- cycle ;
    \draw  [color={rgb, 255:red, 0; green, 0; blue, 0 }  ,draw opacity=1 ][line width=0.75]  (340.92,97.33) .. controls (340.92,88.96) and (364.78,82.17) .. (394.21,82.17) .. controls (423.64,82.17) and (447.5,88.96) .. (447.5,97.33) .. controls (447.5,105.71) and (423.64,112.5) .. (394.21,112.5) .. controls (364.78,112.5) and (340.92,105.71) .. (340.92,97.33) -- cycle ;
    \path  [shading=_e9qihdtow,_svf4mjzjm] (528.22,14.11) .. controls (534.45,19.72) and (523.53,41.99) .. (503.84,63.87) .. controls (484.15,85.75) and (463.14,98.94) .. (456.92,93.33) .. controls (450.69,87.73) and (461.61,65.45) .. (481.3,43.58) .. controls (500.99,21.7) and (522,8.51) .. (528.22,14.11) -- cycle ;
    \draw  [color={rgb, 255:red, 0; green, 0; blue, 0 }  ,draw opacity=1 ][line width=0.75]  (528.22,14.11) .. controls (534.45,19.72) and (523.53,41.99) .. (503.84,63.87) .. controls (484.15,85.75) and (463.14,98.94) .. (456.92,93.33) .. controls (450.69,87.73) and (461.61,65.45) .. (481.3,43.58) .. controls (500.99,21.7) and (522,8.51) .. (528.22,14.11) -- cycle ;
    \path  [shading=_bw97u0oo5,_ddhli8fcf] (447.5,97.33) .. controls (447.5,94.46) and (449.83,92.13) .. (452.71,92.13) .. controls (455.58,92.13) and (457.92,94.46) .. (457.92,97.33) .. controls (457.92,100.21) and (455.58,102.54) .. (452.71,102.54) .. controls (449.83,102.54) and (447.5,100.21) .. (447.5,97.33) -- cycle ;
    \draw  [color={rgb, 255:red, 0; green, 0; blue, 0 }  ,draw opacity=1 ][line width=0.75]  (447.5,97.33) .. controls (447.5,94.46) and (449.83,92.13) .. (452.71,92.13) .. controls (455.58,92.13) and (457.92,94.46) .. (457.92,97.33) .. controls (457.92,100.21) and (455.58,102.54) .. (452.71,102.54) .. controls (449.83,102.54) and (447.5,100.21) .. (447.5,97.33) -- cycle ;
    \path  [shading=_irmxrz6bl,_rzri9nu6o] (332.22,101.61) .. controls (338.45,107.22) and (327.53,129.49) .. (307.84,151.37) .. controls (288.15,173.25) and (267.14,186.44) .. (260.92,180.83) .. controls (254.69,175.23) and (265.61,152.95) .. (285.3,131.08) .. controls (304.99,109.2) and (326,96.01) .. (332.22,101.61) -- cycle ;
    \draw  [color={rgb, 255:red, 0; green, 0; blue, 0 }  ,draw opacity=1 ][line width=0.75]  (332.22,101.61) .. controls (338.45,107.22) and (327.53,129.49) .. (307.84,151.37) .. controls (288.15,173.25) and (267.14,186.44) .. (260.92,180.83) .. controls (254.69,175.23) and (265.61,152.95) .. (285.3,131.08) .. controls (304.99,109.2) and (326,96.01) .. (332.22,101.61) -- cycle ;
    \path  [shading=_ria03yqfs,_tv3cr0nbo] (330.5,97.33) .. controls (330.5,94.46) and (332.83,92.13) .. (335.71,92.13) .. controls (338.58,92.13) and (340.92,94.46) .. (340.92,97.33) .. controls (340.92,100.21) and (338.58,102.54) .. (335.71,102.54) .. controls (332.83,102.54) and (330.5,100.21) .. (330.5,97.33) -- cycle ;
    \draw  [color={rgb, 255:red, 0; green, 0; blue, 0 }  ,draw opacity=1 ][line width=0.75]  (330.5,97.33) .. controls (330.5,94.46) and (332.83,92.13) .. (335.71,92.13) .. controls (338.58,92.13) and (340.92,94.46) .. (340.92,97.33) .. controls (340.92,100.21) and (338.58,102.54) .. (335.71,102.54) .. controls (332.83,102.54) and (330.5,100.21) .. (330.5,97.33) -- cycle ;
    \path  [shading=_7fixrohbn,_0hgwncxm6] (252.46,184.96) .. controls (252.46,182.08) and (254.79,179.75) .. (257.67,179.75) .. controls (260.54,179.75) and (262.87,182.08) .. (262.87,184.96) .. controls (262.87,187.83) and (260.54,190.17) .. (257.67,190.17) .. controls (254.79,190.17) and (252.46,187.83) .. (252.46,184.96) -- cycle ;
    \draw  [color={rgb, 255:red, 0; green, 0; blue, 0 }  ,draw opacity=1 ][line width=0.75]  (252.46,184.96) .. controls (252.46,182.08) and (254.79,179.75) .. (257.67,179.75) .. controls (260.54,179.75) and (262.87,182.08) .. (262.87,184.96) .. controls (262.87,187.83) and (260.54,190.17) .. (257.67,190.17) .. controls (254.79,190.17) and (252.46,187.83) .. (252.46,184.96) -- cycle ;
    \path  [shading=_9prsrmsl7,_te42shfqc] (150.95,123.93) .. controls (143.63,128.01) and (126.08,110.48) .. (111.74,84.78) .. controls (97.4,59.08) and (91.7,34.93) .. (99.02,30.85) .. controls (106.33,26.77) and (123.89,44.3) .. (138.23,70) .. controls (152.57,95.7) and (158.26,119.85) .. (150.95,123.93) -- cycle ;
    \draw  [color={rgb, 255:red, 0; green, 0; blue, 0 }  ,draw opacity=1 ][line width=0.75]  (150.95,123.93) .. controls (143.63,128.01) and (126.08,110.48) .. (111.74,84.78) .. controls (97.4,59.08) and (91.7,34.93) .. (99.02,30.85) .. controls (106.33,26.77) and (123.89,44.3) .. (138.23,70) .. controls (152.57,95.7) and (158.26,119.85) .. (150.95,123.93) -- cycle ;
    \path  [shading=_8g14z78sd,_qoggn7fn2] (159.33,130.95) .. controls (163.31,123.58) and (187.53,128.94) .. (213.43,142.93) .. controls (239.32,156.91) and (257.09,174.22) .. (253.11,181.59) .. controls (249.13,188.96) and (224.91,183.6) .. (199.01,169.62) .. controls (173.12,155.63) and (155.35,138.32) .. (159.33,130.95) -- cycle ;
    \draw  [color={rgb, 255:red, 0; green, 0; blue, 0 }  ,draw opacity=1 ][line width=0.75]  (159.33,130.95) .. controls (163.31,123.58) and (187.53,128.94) .. (213.43,142.93) .. controls (239.32,156.91) and (257.09,174.22) .. (253.11,181.59) .. controls (249.13,188.96) and (224.91,183.6) .. (199.01,169.62) .. controls (173.12,155.63) and (155.35,138.32) .. (159.33,130.95) -- cycle ;
    \path  [shading=_hjawm1m3k,_3ytgx3jcr] (154.79,133.61) .. controls (151.92,133.71) and (149.51,131.45) .. (149.42,128.58) .. controls (149.32,125.7) and (151.58,123.3) .. (154.45,123.2) .. controls (157.33,123.11) and (159.73,125.36) .. (159.83,128.24) .. controls (159.92,131.11) and (157.67,133.52) .. (154.79,133.61) -- cycle ;
    \draw  [color={rgb, 255:red, 0; green, 0; blue, 0 }  ,draw opacity=1 ][line width=0.75]  (154.79,133.61) .. controls (151.92,133.71) and (149.51,131.45) .. (149.42,128.58) .. controls (149.32,125.7) and (151.58,123.3) .. (154.45,123.2) .. controls (157.33,123.11) and (159.73,125.36) .. (159.83,128.24) .. controls (159.92,131.11) and (157.67,133.52) .. (154.79,133.61) -- cycle ;
    \path  [shading=_hbtcw454n,_Am47npqyx] (520.88,186.7) .. controls (514.23,191.8) and (494.33,176.98) .. (476.44,153.61) .. controls (458.55,130.24) and (449.43,107.17) .. (456.09,102.08) .. controls (462.74,96.98) and (482.63,111.8) .. (500.53,135.17) .. controls (518.42,158.54) and (527.53,181.61) .. (520.88,186.7) -- cycle ;
    \draw  [color={rgb, 255:red, 0; green, 0; blue, 0 }  ,draw opacity=1 ][line width=0.75]  (520.88,186.7) .. controls (514.23,191.8) and (494.33,176.98) .. (476.44,153.61) .. controls (458.55,130.24) and (449.43,107.17) .. (456.09,102.08) .. controls (462.74,96.98) and (482.63,111.8) .. (500.53,135.17) .. controls (518.42,158.54) and (527.53,181.61) .. (520.88,186.7) -- cycle ;
    \path  [shading=_4ks716t6n,_igkxxgbt5] (225.88,232.5) .. controls (224.09,228.89) and (223.1,224.93) .. (223.1,220.77) .. controls (223.1,204.2) and (238.77,190.77) .. (258.1,190.77) .. controls (277.43,190.77) and (293.1,204.2) .. (293.1,220.77) .. controls (293.1,224.93) and (292.11,228.89) .. (290.32,232.5) -- cycle ;
    \draw   (225.88,232.5) .. controls (224.09,228.89) and (223.1,224.93) .. (223.1,220.77) .. controls (223.1,204.2) and (238.77,190.77) .. (258.1,190.77) .. controls (277.43,190.77) and (293.1,204.2) .. (293.1,220.77) .. controls (293.1,224.93) and (292.11,228.89) .. (290.32,232.5) -- cycle ;
    \draw [color={rgb, 255:red, 0; green, 0; blue, 0 }  ,draw opacity=0.5 ]   (226.38,232.5) -- (233.96,240.33) ;
    \draw [color={rgb, 255:red, 0; green, 0; blue, 0 }  ,draw opacity=0.5 ]   (288.82,232.5) -- (296.41,240.33) ;
    \draw [color={rgb, 255:red, 0; green, 0; blue, 0 }  ,draw opacity=0.5 ]   (231.63,232.5) -- (239.21,240.33) ;
    \draw [color={rgb, 255:red, 0; green, 0; blue, 0 }  ,draw opacity=0.5 ]   (236.63,232.25) -- (244.21,240.08) ;
    \draw [color={rgb, 255:red, 0; green, 0; blue, 0 }  ,draw opacity=0.5 ]   (241.88,232.25) -- (249.46,240.08) ;
    \draw [color={rgb, 255:red, 0; green, 0; blue, 0 }  ,draw opacity=0.5 ]   (246.88,232.25) -- (254.46,240.08) ;
    \draw [color={rgb, 255:red, 0; green, 0; blue, 0 }  ,draw opacity=0.5 ]   (252.13,232.25) -- (259.71,240.08) ;
    \draw [color={rgb, 255:red, 0; green, 0; blue, 0 }  ,draw opacity=0.5 ]   (257.13,232.25) -- (264.71,240.08) ;
    \draw [color={rgb, 255:red, 0; green, 0; blue, 0 }  ,draw opacity=0.5 ]   (262.38,232.25) -- (269.96,240.08) ;
    \draw [color={rgb, 255:red, 0; green, 0; blue, 0 }  ,draw opacity=0.5 ]   (267.88,232.25) -- (275.46,240.08) ;
    \draw [color={rgb, 255:red, 0; green, 0; blue, 0 }  ,draw opacity=0.5 ]   (273.13,232.25) -- (280.71,240.08) ;
    \draw [color={rgb, 255:red, 0; green, 0; blue, 0 }  ,draw opacity=0.5 ]   (278.38,232.25) -- (285.96,240.08) ;
    \draw [color={rgb, 255:red, 0; green, 0; blue, 0 }  ,draw opacity=0.5 ]   (283.63,232.25) -- (291.21,240.08) ;
\end{tikzpicture}
        }}
    \subfloat[Kinematic subtree division.]{
        \resizebox{0.45\textwidth}{!}{
            \tikzset {_j7xuj04w0/.code = {\pgfsetadditionalshadetransform{ \pgftransformshift{\pgfpoint{0 bp } { 0 bp }  }  \pgftransformrotate{-117 }  \pgftransformscale{2 }  }}}
\pgfdeclarehorizontalshading{_9jgwi88kk}{150bp}{rgb(0bp)=(1,1,1);
    rgb(37.5bp)=(1,1,1);
    rgb(50.08184160505022bp)=(0.95,0.95,0.95);
    rgb(57.64583042689732bp)=(0.88,0.88,0.88);
    rgb(61.33184160505022bp)=(0.96,0.96,0.96);
    rgb(100bp)=(0.96,0.96,0.96)}
\tikzset {_i29mdsrh8/.code = {\pgfsetadditionalshadetransform{ \pgftransformshift{\pgfpoint{0 bp } { 0 bp }  }  \pgftransformrotate{-117 }  \pgftransformscale{2 }  }}}
\pgfdeclarehorizontalshading{_4upvn6zpe}{150bp}{rgb(0bp)=(1,1,1);
    rgb(37.5bp)=(1,1,1);
    rgb(50.08184160505022bp)=(0.95,0.95,0.95);
    rgb(57.64583042689732bp)=(0.88,0.88,0.88);
    rgb(61.33184160505022bp)=(0.96,0.96,0.96);
    rgb(100bp)=(0.96,0.96,0.96)}
\tikzset {_qh2wfjpes/.code = {\pgfsetadditionalshadetransform{ \pgftransformshift{\pgfpoint{0 bp } { 0 bp }  }  \pgftransformrotate{-117 }  \pgftransformscale{2 }  }}}
\pgfdeclarehorizontalshading{_feqyochyi}{150bp}{rgb(0bp)=(1,1,1);
    rgb(37.5bp)=(1,1,1);
    rgb(50.08184160505022bp)=(0.95,0.95,0.95);
    rgb(57.64583042689732bp)=(0.88,0.88,0.88);
    rgb(61.33184160505022bp)=(0.96,0.96,0.96);
    rgb(100bp)=(0.96,0.96,0.96)}
\tikzset {_d10v6gqbc/.code = {\pgfsetadditionalshadetransform{ \pgftransformshift{\pgfpoint{0 bp } { 0 bp }  }  \pgftransformrotate{-117 }  \pgftransformscale{2 }  }}}
\pgfdeclarehorizontalshading{_gk602o946}{150bp}{rgb(0bp)=(1,1,1);
    rgb(37.5bp)=(1,1,1);
    rgb(50.08184160505022bp)=(0.95,0.95,0.95);
    rgb(57.64583042689732bp)=(0.88,0.88,0.88);
    rgb(61.33184160505022bp)=(0.96,0.96,0.96);
    rgb(100bp)=(0.96,0.96,0.96)}
\tikzset {_8lna51u6m/.code = {\pgfsetadditionalshadetransform{ \pgftransformshift{\pgfpoint{0 bp } { 0 bp }  }  \pgftransformrotate{-117 }  \pgftransformscale{2 }  }}}
\pgfdeclarehorizontalshading{_l87ke29nk}{150bp}{rgb(0bp)=(1,1,1);
    rgb(37.5bp)=(1,1,1);
    rgb(50.08184160505022bp)=(0.95,0.95,0.95);
    rgb(57.64583042689732bp)=(0.88,0.88,0.88);
    rgb(61.33184160505022bp)=(0.96,0.96,0.96);
    rgb(100bp)=(0.96,0.96,0.96)}
\tikzset {_flvzdphcz/.code = {\pgfsetadditionalshadetransform{ \pgftransformshift{\pgfpoint{0 bp } { 0 bp }  }  \pgftransformrotate{-117 }  \pgftransformscale{2 }  }}}
\pgfdeclarehorizontalshading{_5rkeipgwd}{150bp}{rgb(0bp)=(1,1,1);
    rgb(37.5bp)=(1,1,1);
    rgb(50.08184160505022bp)=(0.95,0.95,0.95);
    rgb(57.64583042689732bp)=(0.88,0.88,0.88);
    rgb(61.33184160505022bp)=(0.96,0.96,0.96);
    rgb(100bp)=(0.96,0.96,0.96)}
\tikzset {_lvfzf5nu3/.code = {\pgfsetadditionalshadetransform{ \pgftransformshift{\pgfpoint{0 bp } { 0 bp }  }  \pgftransformrotate{-117 }  \pgftransformscale{2 }  }}}
\pgfdeclarehorizontalshading{_65i61e978}{150bp}{rgb(0bp)=(1,1,1);
    rgb(37.5bp)=(1,1,1);
    rgb(50.08184160505022bp)=(0.95,0.95,0.95);
    rgb(57.64583042689732bp)=(0.88,0.88,0.88);
    rgb(61.33184160505022bp)=(0.96,0.96,0.96);
    rgb(100bp)=(0.96,0.96,0.96)}
\tikzset {_27lop19l0/.code = {\pgfsetadditionalshadetransform{ \pgftransformshift{\pgfpoint{0 bp } { 0 bp }  }  \pgftransformrotate{-117 }  \pgftransformscale{2 }  }}}
\pgfdeclarehorizontalshading{_h3pg9rf15}{150bp}{rgb(0bp)=(1,1,1);
    rgb(37.5bp)=(1,1,1);
    rgb(50.08184160505022bp)=(0.95,0.95,0.95);
    rgb(57.64583042689732bp)=(0.88,0.88,0.88);
    rgb(61.33184160505022bp)=(0.96,0.96,0.96);
    rgb(100bp)=(0.96,0.96,0.96)}
\tikzset {_3yz7x95el/.code = {\pgfsetadditionalshadetransform{ \pgftransformshift{\pgfpoint{0 bp } { 0 bp }  }  \pgftransformrotate{-117 }  \pgftransformscale{2 }  }}}
\pgfdeclarehorizontalshading{_Ae1dgwonr}{150bp}{rgb(0bp)=(1,1,1);
    rgb(37.5bp)=(1,1,1);
    rgb(50.08184160505022bp)=(0.95,0.95,0.95);
    rgb(57.64583042689732bp)=(0.88,0.88,0.88);
    rgb(61.33184160505022bp)=(0.96,0.96,0.96);
    rgb(100bp)=(0.96,0.96,0.96)}
\tikzset {_hfu6x8eun/.code = {\pgfsetadditionalshadetransform{ \pgftransformshift{\pgfpoint{0 bp } { 0 bp }  }  \pgftransformrotate{-117 }  \pgftransformscale{2 }  }}}
\pgfdeclarehorizontalshading{_titklq3j6}{150bp}{rgb(0bp)=(1,1,1);
    rgb(37.5bp)=(1,1,1);
    rgb(50.08184160505022bp)=(0.95,0.95,0.95);
    rgb(57.64583042689732bp)=(0.88,0.88,0.88);
    rgb(61.33184160505022bp)=(0.96,0.96,0.96);
    rgb(100bp)=(0.96,0.96,0.96)}
\tikzset{every picture/.style={line width=0.75pt}}
\begin{tikzpicture}[x=0.75pt,y=0.75pt,yscale=-1,xscale=1]
    \path  [shading=_9jgwi88kk,_j7xuj04w0] (382.42,149.33) .. controls (382.42,140.96) and (406.28,134.17) .. (435.71,134.17) .. controls (465.14,134.17) and (489,140.96) .. (489,149.33) .. controls (489,157.71) and (465.14,164.5) .. (435.71,164.5) .. controls (406.28,164.5) and (382.42,157.71) .. (382.42,149.33) -- cycle ;
    \draw  [color={rgb, 255:red, 0; green, 0; blue, 0 }  ,draw opacity=1 ][line width=0.75]  (382.42,149.33) .. controls (382.42,140.96) and (406.28,134.17) .. (435.71,134.17) .. controls (465.14,134.17) and (489,140.96) .. (489,149.33) .. controls (489,157.71) and (465.14,164.5) .. (435.71,164.5) .. controls (406.28,164.5) and (382.42,157.71) .. (382.42,149.33) -- cycle ;
    \path  [shading=_4upvn6zpe,_i29mdsrh8] (569.22,66.11) .. controls (575.45,71.72) and (564.53,93.99) .. (544.84,115.87) .. controls (525.15,137.75) and (504.14,150.94) .. (497.92,145.33) .. controls (491.69,139.73) and (502.61,117.45) .. (522.3,95.58) .. controls (541.99,73.7) and (563,60.51) .. (569.22,66.11) -- cycle ;
    \draw  [color={rgb, 255:red, 0; green, 0; blue, 0 }  ,draw opacity=1 ][line width=0.75]  (569.22,66.11) .. controls (575.45,71.72) and (564.53,93.99) .. (544.84,115.87) .. controls (525.15,137.75) and (504.14,150.94) .. (497.92,145.33) .. controls (491.69,139.73) and (502.61,117.45) .. (522.3,95.58) .. controls (541.99,73.7) and (563,60.51) .. (569.22,66.11) -- cycle ;
    \path  [shading=_feqyochyi,_qh2wfjpes] (488.5,149.33) .. controls (488.5,146.46) and (490.83,144.13) .. (493.71,144.13) .. controls (496.58,144.13) and (498.92,146.46) .. (498.92,149.33) .. controls (498.92,152.21) and (496.58,154.54) .. (493.71,154.54) .. controls (490.83,154.54) and (488.5,152.21) .. (488.5,149.33) -- cycle ;
    \draw  [color={rgb, 255:red, 0; green, 0; blue, 0 }  ,draw opacity=1 ][line width=0.75]  (488.5,149.33) .. controls (488.5,146.46) and (490.83,144.13) .. (493.71,144.13) .. controls (496.58,144.13) and (498.92,146.46) .. (498.92,149.33) .. controls (498.92,152.21) and (496.58,154.54) .. (493.71,154.54) .. controls (490.83,154.54) and (488.5,152.21) .. (488.5,149.33) -- cycle ;
    \path  [shading=_gk602o946,_d10v6gqbc] (293.22,191.61) .. controls (299.45,197.22) and (288.53,219.49) .. (268.84,241.37) .. controls (249.15,263.25) and (228.14,276.44) .. (221.92,270.83) .. controls (215.69,265.23) and (226.61,242.95) .. (246.3,221.08) .. controls (265.99,199.2) and (287,186.01) .. (293.22,191.61) -- cycle ;
    \draw  [color={rgb, 255:red, 0; green, 0; blue, 0 }  ,draw opacity=1 ][line width=0.75]  (293.22,191.61) .. controls (299.45,197.22) and (288.53,219.49) .. (268.84,241.37) .. controls (249.15,263.25) and (228.14,276.44) .. (221.92,270.83) .. controls (215.69,265.23) and (226.61,242.95) .. (246.3,221.08) .. controls (265.99,199.2) and (287,186.01) .. (293.22,191.61) -- cycle ;
    \path  [shading=_l87ke29nk,_8lna51u6m] (213.46,274.96) .. controls (213.46,272.08) and (215.79,269.75) .. (218.67,269.75) .. controls (221.54,269.75) and (223.87,272.08) .. (223.87,274.96) .. controls (223.87,277.83) and (221.54,280.17) .. (218.67,280.17) .. controls (215.79,280.17) and (213.46,277.83) .. (213.46,274.96) -- cycle ;
    \draw  [color={rgb, 255:red, 0; green, 0; blue, 0 }  ,draw opacity=1 ][line width=0.75]  (213.46,274.96) .. controls (213.46,272.08) and (215.79,269.75) .. (218.67,269.75) .. controls (221.54,269.75) and (223.87,272.08) .. (223.87,274.96) .. controls (223.87,277.83) and (221.54,280.17) .. (218.67,280.17) .. controls (215.79,280.17) and (213.46,277.83) .. (213.46,274.96) -- cycle ;
    \path  [shading=_5rkeipgwd,_flvzdphcz] (111.95,213.93) .. controls (104.63,218.01) and (87.08,200.48) .. (72.74,174.78) .. controls (58.4,149.08) and (52.7,124.93) .. (60.02,120.85) .. controls (67.33,116.77) and (84.89,134.3) .. (99.23,160) .. controls (113.57,185.7) and (119.26,209.85) .. (111.95,213.93) -- cycle ;
    \draw  [color={rgb, 255:red, 0; green, 0; blue, 0 }  ,draw opacity=1 ][line width=0.75]  (111.95,213.93) .. controls (104.63,218.01) and (87.08,200.48) .. (72.74,174.78) .. controls (58.4,149.08) and (52.7,124.93) .. (60.02,120.85) .. controls (67.33,116.77) and (84.89,134.3) .. (99.23,160) .. controls (113.57,185.7) and (119.26,209.85) .. (111.95,213.93) -- cycle ;
    \path  [shading=_65i61e978,_lvfzf5nu3] (120.33,220.95) .. controls (124.31,213.58) and (148.53,218.94) .. (174.43,232.93) .. controls (200.32,246.91) and (218.09,264.22) .. (214.11,271.59) .. controls (210.13,278.96) and (185.91,273.6) .. (160.01,259.62) .. controls (134.12,245.63) and (116.35,228.32) .. (120.33,220.95) -- cycle ;
    \draw  [color={rgb, 255:red, 0; green, 0; blue, 0 }  ,draw opacity=1 ][line width=0.75]  (120.33,220.95) .. controls (124.31,213.58) and (148.53,218.94) .. (174.43,232.93) .. controls (200.32,246.91) and (218.09,264.22) .. (214.11,271.59) .. controls (210.13,278.96) and (185.91,273.6) .. (160.01,259.62) .. controls (134.12,245.63) and (116.35,228.32) .. (120.33,220.95) -- cycle ;
    \path  [shading=_h3pg9rf15,_27lop19l0] (115.79,223.61) .. controls (112.92,223.71) and (110.51,221.45) .. (110.42,218.58) .. controls (110.32,215.7) and (112.58,213.3) .. (115.45,213.2) .. controls (118.33,213.11) and (120.73,215.36) .. (120.83,218.24) .. controls (120.92,221.11) and (118.67,223.52) .. (115.79,223.61) -- cycle ;
    \draw  [color={rgb, 255:red, 0; green, 0; blue, 0 }  ,draw opacity=1 ][line width=0.75]  (115.79,223.61) .. controls (112.92,223.71) and (110.51,221.45) .. (110.42,218.58) .. controls (110.32,215.7) and (112.58,213.3) .. (115.45,213.2) .. controls (118.33,213.11) and (120.73,215.36) .. (120.83,218.24) .. controls (120.92,221.11) and (118.67,223.52) .. (115.79,223.61) -- cycle ;
    \path  [shading=_Ae1dgwonr,_3yz7x95el] (561.88,238.7) .. controls (555.23,243.8) and (535.33,228.98) .. (517.44,205.61) .. controls (499.55,182.24) and (490.43,159.17) .. (497.09,154.08) .. controls (503.74,148.98) and (523.63,163.8) .. (541.53,187.17) .. controls (559.42,210.54) and (568.53,233.61) .. (561.88,238.7) -- cycle ;
    \draw  [color={rgb, 255:red, 0; green, 0; blue, 0 }  ,draw opacity=1 ][line width=0.75]  (561.88,238.7) .. controls (555.23,243.8) and (535.33,228.98) .. (517.44,205.61) .. controls (499.55,182.24) and (490.43,159.17) .. (497.09,154.08) .. controls (503.74,148.98) and (523.63,163.8) .. (541.53,187.17) .. controls (559.42,210.54) and (568.53,233.61) .. (561.88,238.7) -- cycle ;
    \draw [color={rgb, 255:red, 74; green, 144; blue, 226 }  ,draw opacity=1 ][line width=1.5]    (289.92,196.67) -- (384.71,151.46) ;
    \draw [shift={(387.42,150.17)}, rotate = 154.5] [color={rgb, 255:red, 74; green, 144; blue, 226 }  ,draw opacity=1 ][line width=1.5]    (17.05,-5.13) .. controls (10.84,-2.18) and (5.16,-0.47) .. (0,0) .. controls (5.16,0.47) and (10.84,2.18) .. (17.05,5.13)   ;
    \draw  [color={rgb, 255:red, 74; green, 144; blue, 226 }  ,draw opacity=1 ][dash pattern={on 4.5pt off 4.5pt}] (366.42,149.33) .. controls (366.42,92.95) and (428.2,47.25) .. (504.42,47.25) .. controls (580.63,47.25) and (642.42,92.95) .. (642.42,149.33) .. controls (642.42,205.71) and (580.63,251.42) .. (504.42,251.42) .. controls (428.2,251.42) and (366.42,205.71) .. (366.42,149.33) -- cycle ;
    \path  [shading=_titklq3j6,_hfu6x8eun] (188.28,322.23) .. controls (186.49,318.63) and (185.5,314.66) .. (185.5,310.5) .. controls (185.5,293.93) and (201.17,280.5) .. (220.5,280.5) .. controls (239.83,280.5) and (255.5,293.93) .. (255.5,310.5) .. controls (255.5,314.66) and (254.51,318.63) .. (252.72,322.23) -- cycle ;
    \draw   (188.28,322.23) .. controls (186.49,318.63) and (185.5,314.66) .. (185.5,310.5) .. controls (185.5,293.93) and (201.17,280.5) .. (220.5,280.5) .. controls (239.83,280.5) and (255.5,293.93) .. (255.5,310.5) .. controls (255.5,314.66) and (254.51,318.63) .. (252.72,322.23) -- cycle ;
    \draw [color={rgb, 255:red, 0; green, 0; blue, 0 }  ,draw opacity=0.5 ]   (188.78,322.23) -- (196.36,330.06) ;
    \draw [color={rgb, 255:red, 0; green, 0; blue, 0 }  ,draw opacity=0.5 ]   (251.22,322.23) -- (258.81,330.06) ;
    \draw [color={rgb, 255:red, 0; green, 0; blue, 0 }  ,draw opacity=0.5 ]   (194.03,322.23) -- (201.61,330.06) ;
    \draw [color={rgb, 255:red, 0; green, 0; blue, 0 }  ,draw opacity=0.5 ]   (199.03,321.98) -- (206.61,329.81) ;
    \draw [color={rgb, 255:red, 0; green, 0; blue, 0 }  ,draw opacity=0.5 ]   (204.28,321.98) -- (211.86,329.81) ;
    \draw [color={rgb, 255:red, 0; green, 0; blue, 0 }  ,draw opacity=0.5 ]   (209.28,321.98) -- (216.86,329.81) ;
    \draw [color={rgb, 255:red, 0; green, 0; blue, 0 }  ,draw opacity=0.5 ]   (214.53,321.98) -- (222.11,329.81) ;
    \draw [color={rgb, 255:red, 0; green, 0; blue, 0 }  ,draw opacity=0.5 ]   (219.53,321.98) -- (227.11,329.81) ;
    \draw [color={rgb, 255:red, 0; green, 0; blue, 0 }  ,draw opacity=0.5 ]   (224.78,321.98) -- (232.36,329.81) ;
    \draw [color={rgb, 255:red, 0; green, 0; blue, 0 }  ,draw opacity=0.5 ]   (230.28,321.98) -- (237.86,329.81) ;
    \draw [color={rgb, 255:red, 0; green, 0; blue, 0 }  ,draw opacity=0.5 ]   (235.53,321.98) -- (243.11,329.81) ;
    \draw [color={rgb, 255:red, 0; green, 0; blue, 0 }  ,draw opacity=0.5 ]   (240.78,321.98) -- (248.36,329.81) ;
    \draw [color={rgb, 255:red, 0; green, 0; blue, 0 }  ,draw opacity=0.5 ]   (246.03,321.98) -- (253.61,329.81) ;
    \draw (317.5,149.4) node [anchor=north west][inner sep=0.75pt]  [color={rgb, 255:red, 74; green, 144; blue, 226 }  ,opacity=1 ]  {${\displaystyle \mathbf{f_{i}}}$};
    \draw (429.5,140.9) node [anchor=north west][inner sep=0.75pt]    {$\textcolor[rgb]{0.29,0.56,0.89}{i}$};
    \draw  [draw opacity=0]  (256.6, 230.8) circle [x radius= 21.92, y radius= 14.85]   ;
    \draw (242.1,222.7) node [anchor=north west][inner sep=0.75pt]    {$\textcolor[rgb]{0.29,0.56,0.89}{\lambda }\textcolor[rgb]{0.29,0.56,0.89}{(}\textcolor[rgb]{0.29,0.56,0.89}{i}\textcolor[rgb]{0.29,0.56,0.89}{)}$};
\end{tikzpicture}
        }}
\end{figure}

The \textit{Articulated-Body Algorithm} \citep{featherstone_rigid_2008} finds its roots by initially considering a subdivision of the kinematic structure in sub-trees as shown in \cref{fig:kin_tree}, i.e. starting from the leaves and going down to the root. If we consider a joint $i$ interacting with the rest of the kinematic chain, it will be subject to an unknown force $\mathbf{f} _i$ defined as:

\begin{equation}
    \label{eqn:biasforce}
    \mathbf{f} _i = \mathbb{M} _i ^A \mathbf{a} _i + \mathbf{p} ^A _i
\end{equation}

where $\mathbb{M} _i ^A$ is the articulated body inertia, i.e. the inertia expressed at the base of the subtree when all its joints are fixed, and $\mathbf{p} ^A _i$ is the associated bias force.

In ABA, first, a forward pass is performed to compute the initial articulated body inertia, the initial bias forces, and the link velocities, then a backward pass is performed to finalize the computation of the articulated body inertia and the bias force at each joint and finally a forward pass computes the acceleration of each body in the kinematic chain.

By indicating with $\lambda(i)$ the parent link of link $i$, we can express the spatial force acting and the acceleration of body $i$:

\begin{equation}
    \label{eqn:aba_torque}
    \boldsymbol{\tau} _i = \boldsymbol{\Phi} ^\top _i \mathbf{f} _i = \boldsymbol{\Phi} ^\top _i (\mathbb{M} _i ^A \mathbf{a} _i + \mathbf{p} ^A _i) = \boldsymbol{\Phi} ^\top _i (\mathbb{M} _i ^A (\mathbf{a} _{\lambda(i)} + \boldsymbol{\Phi} _i \ddot{\mathbf{s}} _i + \partial_i \boldsymbol{\Phi} _i \dot{\mathbf{s}} _i)+ \mathbf{p} ^A _i)
\end{equation}

where $\boldsymbol{\Phi} _i$ is the motion subspace of the $i$-th link, $\mathbf{a} _i$ is its acceleration, $\mathbf{a} _{\lambda(i)}$ is the acceleration of the parent link, $\ddot{\mathbf{s}} _i$ is the acceleration of the $i$-th joint, and $\partial_i \boldsymbol{\Phi} _i$ is the derivative of the motion subspace of the $i$-th link with respect to the joint coordinates.
By solving \cref{eqn:aba_torque} for $\ddot{\mathbf{s}}$ the following results is obtained:

\begin{equation}
    \ddot{\mathbf{s}} _i = \mathbb{M} _i ^{-1} (\boldsymbol{\tau} _i - \boldsymbol{\Phi} ^\top _i \mathbf{p} ^A _i - \boldsymbol{\Phi} ^\top _i \mathbb{M} _i ^A \mathbf{a} _{\lambda(i)})
\end{equation}

This yields that once the articulated body inertia and the bias force are known, it is possible to compute the acceleration of each joint independently from the dynamics of the other joints, constituting the main advantage of the articulated-body algorithm.

\begin{algorithm}[h]
    \caption{Articulated-Body Algorithm.}
    \label{alg:aba}
    \begin{algorithmic}[1]
        \Require $\mathcal{M}, \mathbf{s}, \dot{\mathbf{s}}, \boldsymbol{\tau}$

        \For{$i = 1 \text{ to } n$}
        \Comment{Backward Pass}
        \State $[\mathbf{X}_J, \boldsymbol{\Phi}_i] = \text{jcalc}(\text{jtype}(i), \dot{\mathbf{s}}_i)$
        \State $\mathrm{\mathbf{v}}_J = \boldsymbol{\Phi}_i \dot{\mathbf{s}}_i$
        \State $^i\mathbf{X}_{\lambda(i)} = \mathbf{X}_J\mathbf{X}_T (i)$
        \If{$\lambda_i = 0$}
        \State $\mathrm{\mathbf{v}}_i = \mathrm{\mathbf{v}}_J$
        \State $\mathbf{c}_i = \mathbf{0}$
        \Else
        \State $\mathrm{\mathbf{v}}_i = {}^i\mathbf{X} _{\lambda(i)}\mathrm{\mathbf{v}}_{\lambda(i)} + \mathrm{\mathbf{v}}_J$
        \State $\mathbf{c}_i = \mathrm{\mathbf{v}}_i \times ^* \mathrm{\mathbf{v}}_J$
        \EndIf
        \State $\mathbb{M}_i ^A = \mathbb{M}_i$
        \State $\mathbf{p}_i ^A = \mathrm{\mathbf{v}}_i \times^* \mathbb{M}_i ^A \mathrm{\mathbf{v}}_i - ^i\mathbf{X} _0 ^* f ^* _i $
        \EndFor

        \item[]

        % Pass 2
        \For{$i = n \text{ to } 1$}
        \Comment{Forward Pass}
        \State $\mathbf{U}_i = \mathbb{M}_i ^A \boldsymbol{\Phi}_i$
        \State $\mathbf{D} _i = \boldsymbol{\Phi} ^\top _i  {} \mathbf{U} _i \boldsymbol{\Phi} _i $
        \State $\mathbf{u}_i = \boldsymbol{\tau}_i - \boldsymbol{\Phi}_i\mathbf{p}_i^A$
        \If{$\lambda_i \neq 0$}
        \State $\mathbb{M} ^A = \mathbb{M} ^A _{\lambda (i)} + {} ^i X _{\lambda (i)} ^\top (\mathbb{M} _i ^A - {}  \mathbf{U} _i  \mathbf{D} ^{-1} _i  {}  \mathbf{U} ^\top _i) {} ^i X _{\lambda (i)} $
        \State $\mathbf{p} ^A = \mathbf{p} ^A _{\lambda (i)} + {} ^i X _{\lambda (i)} ^\top (\mathbf{p} ^A_i + \mathbb{M} ^A _{\lambda (i)}  \mathbf{c}_i + {}  \mathbf{U} _i \mathbf{D} ^{-1} _i {} \mathbf{u} _i) $
        \EndIf
        \EndFor

        \item[]

        % Pass 3
        \For{$i = 1 \text{ to } n$}
        \Comment{Backward Pass}
        \If{$\lambda_i = 0$}
        \State $\mathbf{a}' = -\mathbf{a}_g$
        \Else
        \State $\mathbf{a}' = {}^{\lambda(i)}\mathbf{X}_i \mathbf{a}_{\lambda(i)}$
        \State $\ddot{\mathbf{s}}_i = \mathbf{D}^{-1} (\mathbf{u}_i - (\mathbf{U}_i^\top)\mathbf{a}')$
        \State $\mathbf{a}_i = \mathbf{a}' + \boldsymbol{\Phi}_i\mathbf{\ddot{q}}_i + \mathbf{c} _i$
        \EndIf
        \EndFor
    \end{algorithmic}
\end{algorithm}
