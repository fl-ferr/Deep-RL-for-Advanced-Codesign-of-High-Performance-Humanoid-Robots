\chapter*{Epilogue}
\label{chp:99-Epilogue}

This work presented a novel framework based on the combination of reinforcement learning and evolutionary algorithms for the codesign of humanoid robots supported by hardware accelerated simulators. In particular, it answers the research question:

\begin{quote}
    \textit{
        Can humanoid robots be endowed with embodied cognition?}
\end{quote}

in which \textit{embodied cognition} refers to a synergy between the robot's hardware and the control system that makes an intelligence behavior emerge. In the scope of this thesis, the hardware is represented by the motors actuating the joints of the robot, and the control system is represented by a control policy obtained via deep reinforcement learning.
In order to answer this question, the work presents a novel framework that combines reinforcement learning and evolutionary algorithms to optimize the motor parameters of a robot with respect to a given task. In particular,

\begin{description}
    \item[In \cref{part:background}] the work presented the background knowledge required to understand the work, starting from the notation and the mathematical preliminaries of rigid multibody dynamics. The groundwork was then used to present the original formulation of the \ac{ABA} algorithm, which is the basis for the forward dynamics computation in this framework. Then, we moved to an overview of the current state of the art in the scope of hardware accelerated simulators, with a focus on the emerging role of \jax in the field of deep learning and robotics. Finally, the theoretical background of deep reinforcement learning and evolutionary algorithms was presented, introducing the reader to the main concepts of the two fields and the most common algorithms used in the literature.
    \item[In \cref{part:contributions}] the thesis proposed the two main contributions of the thesis. First, it guided the reader to the understanding of the implementation of a modified version of the \ac{ABA} algorithm that allows a holistic computation of the forward dynamics of a multibody system when there is the need to take into account the motor dynamics. Moreover, the reformulation served as an additional feature for the hardware accelerated \jaxsim simulator, allowing us to consider the effect of rotors in the robot joints during simulation. As a second contribution, the thesis described the implementation of a codesign loop that creates a synergy between reinforcement learning and evolutionary algorithms in order to guide the optimization of the motor parameters of a robot, while learning a policy able to act as its control system.
\end{description}

This work addressed therefore the problem of the limited offer of simulation environments that allow for a computation of the forward dynamics that takes into account the presence of motors mounted on the joints of the robot in a hardware accelerated simulator. The framework was tested on a simple, yet easily scalable environment, and the results showed that the developed architecture is able to find a set of motor parameters that allows for a better reward in the reinforcement learning scenario.
The proposed approach for the codesign can potentially be a starting point for a further extension to the generation of new humanoid robots in which the hardware is optimized together with the control strategy. Furthermore, it could be propelled by the recent advances in the field of hardware acceleration and the development of differentiable physics engines, which is crucial for the development of a more extensive co-optimization architecture.
