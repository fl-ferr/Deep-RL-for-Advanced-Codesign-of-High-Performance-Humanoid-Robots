\chapter*{Epilogue}
\label{chp:99-Epilogue}

This work presented a novel framework based on the combination of reinforcement learning and evolutionary algorithms for the codesign humanoid robots. In particular, it answers the research question:

\begin{quote}
    \textit{
        How can a set of parameter values be found that optimizes the performance of a humanoid robot in a given task?}
\end{quote}

and in particular, it addresses the problem of finding the optimal motor parameters for the locomotion task. It addressed also the problem of the lack of a simulation environment that allows for a computation of the forward dynamics that takes into account the presence of motors mounted on the joints of the robot, allowing for a potentially smaller gap between the simulation and the real world application.
The framework was tested on a simple, yet easily scalable environment, and the results showed that the proposed framework is able to find a set of motor parameters that allows for a better reward in the reinforcement learning framework.
The proposed framework is could be propelled by the recent advances in the field of hardware acceleration.

\CarlottaSays{add also the other contribution, the usage of hardware accelerated, the usage of deep rl for codesign, the codesing in itself.}