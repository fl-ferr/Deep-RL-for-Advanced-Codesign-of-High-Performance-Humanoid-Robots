\chapter*{Prologue}
\label{chp:00-Prologue}

\section*{Motivation and Objectives}

Robots are often associated with their capacity to move and interact with the environment, known as motion intelligence \CarlottaSays{This motion intelligence here is a little bit out of the blue, should be better specified (it is also useful to introduce such a concept already ? )}. We picture robots navigating through complex  \CarlottaSays{spaces Fishy word}, picking up objects, and performing various tasks. However, a critical aspect that is sometimes overlooked is the physical design of the robot's body. The two fundamental objectives in the framework of robotics are achieving energy efficiency and high performance. Energy efficiency involves a robot's ability to carry out tasks using minimal energy, which is crucial for battery-powered mobile robots. Performance, on the other hand, pertains to how effectively the robot can accomplish its intended tasks, exploiting its morphology and the environment to its advantage. Thinking of a robot's hardware as its body, dictates the ability to perform tasks and the energy consumption, just as in car design the materials used, the physical shape and size, impact its speed, fuel efficiency, and maneuverability.

The morphological and physical properties of a robot's hardware are often neglected, as mainly provided by the manufacturer, and the research is focused on the development of the robot's motion intelligence. However, the robot's hardware plays a pivotal role in achieving energy efficiency and high performance.
Sometimes, during the development of robots, particularly regarding their control systems, the physical aspects of the robot's body are treated as unalterable components. In such cases, the design of the robot's software and algorithms may not consider how the body can be optimized for specific tasks, potentially leading to inefficiencies and limitations in the robot's performance. Conversely, when engineers design the physical structure of the robot, they may not always take into account the specific tasks the robot is meant to perform. The lack of alignment between the robot's hardware and its intended functions can result in suboptimal designs that hinder the robot's efficiency and effectiveness in real-world applications.


When dealing with humanoid robotics, the intricacy of feature engineering is often time-consuming, hardly flexible, and might lead to sub-optimal assets. Furthermore, the complexity given by the realistic multibody dynamics involving friction, contacts, elasticity, and gravity might be restricting when there is a need for the robot to adapt to new scenarios.

In recent years, the field of robotics has been propelled by the combination of artificial intelligence and mechanical design. Such fusion pushed the frontiers of applications that led to an inexorable increase in the need to face progressively harder challenges and tasks. However, it is the application of deep reinforcement learning that has truly unlocked their potential, propelling them into realms of unprecedented intelligence, adaptability, and interaction.

Picture a robot capable of physically interacting with objects, learning complex manipulation skills that mimic human dexterity, walking robustly, and effortlessly adapting to novel scenarios without explicit programming. These visions are rapidly becoming a reality, thanks to the convergence of deep \ac{RL} and humanoid robot codesign.

In essence, achieving optimal performance and energy efficiency in robotics requires a holistic approach. It's not just about making a robot smart but also crafting a body that aligns with its intended tasks. By considering both the robot's motion intelligence and hardware together, we can create robots that are not only intelligent but also proficient and economical in their actions.

The work aims to provide a comprehensive understanding of the state of the art in this rapidly evolving field, identify potential avenues for future research, and contribute to innovation by exploring a new path of embodied intelligence and reinforcement learning applied to humanoid robot codesign.

\section*{Contributions}

The main contributions of this work can be summarized as follows:

\begin{enumerate}
    \item A comprehensive review of the state of the art in the field of humanoid robot codesign and reinforcement learning.
    \item An in-depth analysis and revamping of a differentiable physics simulator, with a focus on contact dynamics and forward dynamics computation.
    \item A novel formulation for the computation of forward dynamics in fast and effective recursive rigid body dynamics algorithms that includes the contributions of motor dynamics.
    \item An optimization framework for the codesign of humanoid robots, that exploits the potential of deep reinforcement learning and genetic algorithms.
    \item A set of experiments that demonstrate the effectiveness of the proposed framework.
\end{enumerate}


\subsection*{Outline}

The present work is organized as follows:

\begin{description}

    \item{\hyperref[part:background]{Part I: Background}}

          \begin{description}
              \item[{\hyperref[chp:back_RBDynamics]{In the first chapter}}] The mathematical foundations of rigid body dynamics are presented, along with the notation and the conventions that are used throughout the work.
              \item[{\hyperref[chp:back_PhysicsSimulators]{In the second chapter}}] The current panorama of physics simulators is presented, with a focus on the differentiable simulators and in particular on the one exploited and modified for the purpose of this work.
              \item [{\hyperref[chp:back_RLGA]{In the third chapter}}] The fundamentals of reinforcement learning and evolutionary algorithms are presented, with a focus on the algorithms and the techniques that are exploited in the codesign loop.
          \end{description}

    \item{\hyperref[part:contributions]{Part II: Contributions}}

          \begin{description}
              \item[{\hyperref[chp:contrib_ABA]{In the fourth chapter}}] The implementation of a recursive rigid body dynamics algorithm that takes into account motor dynamics is presented.
              \item[{\hyperref[chp:contrib_JaxSim]{In the fifth chapter}}] The differentiable physics simulator is presented, along with the modifications and the improvements that have been implemented.
              \item[{\hyperref[chp:contrib_CodesignRL]{In the sixth chapter}}] The state of the art in the field of codesign and reinforcement learning is presented, furthermore the methods and the main challenges of the implementation of a complete codesign framework that exploits the potential of reinforcement learning and genetic algorithms are discussed.
              \item[{\hyperref[chp:contrib_ResultsDiscussion]{In the seventh chapter}}] The results of the experiments are presented and discussed, with a focus on the analysis of the performance of the different algorithms and the comparison between the different approaches.
              \item[{\hyperref[chp:contrib_Conclusions]{In the last chapter}}] The conclusions of the present work are drawn, with an eye on future developments and the potential applications of the proposed framework.
          \end{description}
\end{description}