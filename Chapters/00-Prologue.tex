\chapter*{Prologue}
\label{chp:00-Prologue}

\section*{Motivation and Objectives}

Robots are often associated with their capacity to move and interact with the environment, navigating through complex environments, picking up objects, and performing various tasks in a wide range of applications. The field of robotics has been rapidly evolving in recent years, with robots becoming increasingly aware of their surroundings and behaving accordingly, eventually becoming more capable of performing complex tasks. This process, known as motion intelligence or embodied cognition, involves the development of algorithms that enable robots to perform tasks autonomously and receive and process information coming from the environment and most of all from their own body. The most common approach to achieve this is to use a combination of sensors and algorithms to enable the robot to perceive its movements and make decisions based on that data \citep{metta_icub_2008}.

However, the morphological and physical properties of a robot's hardware are sometimes overlooked during the development of these control systems, as mainly provided by the manufacturer, and the research is focused on the development of the robot's motion intelligence. However, the robot's hardware plays a pivotal role in achieving energy efficiency and effectiveness for a set of tasks.
In such cases, the design of the robot's software and algorithms may not consider how the body can be optimized for specific tasks, potentially leading to inefficiencies and limitations in the robot's performance. Conversely, when engineers design the physical structure of the robot, they may not always take into account the specific tasks the robot is meant to perform. The lack of alignment between the robot's hardware and its intended functions can result in suboptimal designs that hinder the robot's performance in real-world applications \citep{vaisi_review_2022, sartore_optimization_2022}.

Thus, taking a comprehensive perspective on the robot's performance and energy efficiency requires an in-depth examination of both its hardware and control system. This approach can be crucial because it involves leveraging the inherent characteristics of the robot's physical design to optimize its efficiency and effectiveness for a set of given tasks. By doing so, we can substantially reduce the forces and torques required for task completion, a particularly critical aspect for battery-powered robots. Furthermore, this holistic approach aims to not only minimize energy consumption but also to maximize the robot's performance metrics, including speed, accuracy, and robustness, ensuring that it operates at peak efficiency.
With this new approach, the robot's body is not just a passive component but an active part of the control system loop, and the control system is not just a set of algorithms but a key parameter in the robot morphology design. This concept is commonly referred to as \textit{codesign}.

When dealing with humanoid robotics, the intricacy of feature engineering is often time-consuming, hardly flexible, and might lead to sub-optimal assets. Furthermore, the complexity given by the realistic multibody dynamics involving friction, contacts, elasticity, and gravity might lead to poor results when there is a need for the robot to adapt to new scenarios. This combination of factors makes it challenging to design a robot that can perform a wide range of tasks in a variety of environments and using classical optimization approaches might become infeasible. In recent years, the field of robotics has been propelled by the combination of artificial intelligence and mechanical design \citep{fadini_simulation_2022}. Such fusion pushed the frontiers of applications that led to a rapid increase in the need to face progressively harder challenges and tasks.
Picture a robot capable of physically interacting with objects, learning complex and robust human-like walking, and effortlessly adapting to novel scenarios without explicit design for those new tasks.
These visions require a holistic approach to the design of the robot, where there is no need to explicitly formulate the dynamics of the robot, but rather to let the robot learn how to move and interact with the environment by itself using the available sensors and actuators. That is why the application of deep reinforcement learning (\ac{RL}) to robotics has been gaining momentum in recent years \citep{golroudbari_recent_2023,li_reinforcement_2021}.

Previous attempts at codesign for robotics have been made considering an environment with limited complexity, such as a 2D plane \citep{ha_reinforcement_2019}, optimizing simplified hardware with a limited number of links \citep{chen_hardware_2020} or avoiding retraining the neural network (\ac{NN})-model at each optimizer iteration \citep{bjelonic_learning-based_2023}, potentially leading to a sub-optimality propagation throughout the whole design process. Though taking into account the whole-body dynamics of a humanoid robot can result in a more realistic and accurate model, it requires high computational power and time, making it difficult to use in a codesign loop. That is why it has not been widely investigated in the field of codesign for robotics. Yet, with the use of hardware acceleration, it is possible to optimize the computational effort required reducing the time needed for the simulation.

In light of the limitations identified in previous codesign efforts, which often operated within simplified environments and models, this work aims to explore a novel approach, trying to overcome the challenges of codesign by exploiting the capabilities of modern hardware-accelerated architectures as a support of a codesign loop combining genetic algorithms (\ac{GA}) and deep reinforcement learning.

% In essence, achieving optimal performance and energy efficiency in robotics requires a holistic approach. It's not just about making a robot smart but also crafting a body that aligns with its intended tasks. By considering both the robot control system and hardware together, we can create robots that are not only intelligent but also proficient in their actions.


\section*{Contributions}

The main contributions of this work can be summarized as follows:

\begin{enumerate}
    \item An in-depth analysis and enhancement of a differentiable physics simulator, with a focus on contact dynamics and forward dynamics computation.
    \item A motor dynamics conditioned formulation for the forward dynamics of a rigid multibody system with the use of recursive propagation methods.
    \item A co-optimization framework for the codesign of humanoid robots, that exploits the potential of deep reinforcement learning and genetic algorithms, supported by hardware acceleration.
    \item A set of experiments that demonstrate the effectiveness and the limitations of the proposed framework.
\end{enumerate}


\subsection*{Outline}

The present work is organized as follows:

\begin{description}

    \item{\hyperref[part:background]{Part I: Background}}

          \begin{description}
              \item[{\hyperref[chp:back_RBDynamics]{In the first chapter}}] The mathematical foundations of rigid body dynamics are presented, along with the notation and the conventions that are used throughout the work.
              \item[{\hyperref[chp:back_PhysicsSimulators]{In the second chapter}}] The current panorama of physics simulators is presented, with a focus on the differentiable simulators and in particular on the one exploited and modified for the purpose of this work.
              \item [{\hyperref[chp:back_RLGA]{In the third chapter}}] The fundamentals of reinforcement learning and evolutionary algorithms are presented, with a focus on the algorithms and the techniques that are exploited in the codesign loop.
          \end{description}

    \item{\hyperref[part:contributions]{Part II: Contributions}}

          \begin{description}
              \item[{\hyperref[chp:contrib_ABA]{In the fourth chapter}}] The implementation of a recursive rigid body dynamics algorithm that takes into account motor dynamics is presented, along with the details regarding its implementation in a hardware-accelerated simulator.
              \item[{\hyperref[chp:contrib_CodesignRL]{In the fifth chapter}}] The state of the art in the field of codesign and reinforcement learning is presented, furthermore the methods and the main challenges of the implementation of a complete codesign framework that exploits the potential of reinforcement learning and genetic algorithms are discussed.
              \item[{\hyperref[chp:contrib_ResultsDiscussion]{In the sixth chapter}}] The results of the experiments are presented and discussed, with a focus on the analysis of the performance of the different algorithms and the comparison between the different approaches.
              \item[{\hyperref[chp:contrib_Conclusions]{In the last chapter}}] The conclusions of the present work are drawn, with an eye on future developments and the potential applications of the proposed framework.
          \end{description}
\end{description}