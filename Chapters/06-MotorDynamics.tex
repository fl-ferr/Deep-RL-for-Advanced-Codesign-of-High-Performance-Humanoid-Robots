\chapter{Implementing Motors in Rigid Multibody Algorithms}
\label{chp:06-MotDyn}

\section{Problem Formalization}

Starting from the equation of motion of a robot manipulator:

$$
    \mathbf{M}(q)\dot{\boldsymbol{\nu}} + \mathbf{h}(q,\boldsymbol{\nu}) = \mathbf{B}\boldsymbol{\tau} + \mathbf{J} ^T \mathbf{f}
$$

where:

\begin{itemize}
    \item $\mathbf{M}(q)$ is the inertia matrix
    \item $\mathbf{h}(q,\boldsymbol{\nu})$ is the vector of Coriolis, centrifugal and gravitational forces
    \item $\mathbf{B}$ is the matrix of actuation
    \item $\boldsymbol{\tau}$ is the vector of actuation torques
    \item $\mathbf{J}$ is the Jacobian matrix
    \item $\mathbf{f}$ is the vector of the external forces
\end{itemize}

we can isolate the terms related to the base link (usually in position 0) from the joints' poses:

\begin{align}
    \boldsymbol{\nu} =
    \begin{bmatrix}
        \mathrm{\mathbf{v}} \\
        \dot{\mathbf{s}}
    \end{bmatrix} &  &
    \dot{\boldsymbol{\nu}} =
    \begin{bmatrix}
        \dot{\mathrm{\mathbf{v}}} \\
        \ddot{\mathbf{s}}
    \end{bmatrix}
\end{align}

where $\mathbb{\mathrm{v}} \in \mathbb{R} ^{6}$ and $\mathbf{s} \in \mathbb{R}^{N_B}$, we get to the form:

\begin{equation}
    \begin{bmatrix}
        \mathbf{M} _{\mathcal{B}}(q)     & \mathbf{M} _{\mathcal{B}S}(q) \\
        \mathbf{M} _{\mathcal{B}S} ^T(q) & \mathbf{M} _s(q)
    \end{bmatrix}
    \begin{bmatrix}
        \dot{\mathrm{\mathbf{v}}} \\
        \ddot{\mathbf{s}}
    \end{bmatrix}+
    \begin{bmatrix}
        \mathbf{h} _{\mathcal{B}} \\
        \mathbf{h} _S
    \end{bmatrix}=
    \begin{bmatrix}
        \mathbb{0} \\
        \mathbb{1}
    \end{bmatrix}
    \boldsymbol{\tau}
    +
    \begin{bmatrix}
        \mathbf{J} _{\mathcal{B}} \\
        \mathbf{J} _S
    \end{bmatrix} ^T
    \begin{bmatrix}
        \mathbf{f} _{\mathcal{B}} \\
        \mathbf{f} _S
    \end{bmatrix}
\end{equation}

Given that the dynamics of the set of motors can be described by the following equation:

\begin{equation}
    \label{eqn:mot_dyn}
    \mathbf{I_m} \ddot{\boldsymbol{\theta}} + \mathbf{K}_v \dot{\boldsymbol{\theta}} = \boldsymbol{\tau}_m
\end{equation}

where $\mathbf{K _v}$ is the diagonal matrix of motor viscous coefficients and $\mathbf{I}_m$ is the diagonal matrix of motors' inertias. Considering that given the set of transmission ratios $\boldsymbol{\Gamma}$, the relation between the joints' velocities and the motors' velocities is:

\begin{align}
    \mathbf{s} = \boldsymbol{\theta} \boldsymbol{\Gamma} &  & \dot{\mathbf{s}} = \dot{\boldsymbol{\theta}} \boldsymbol{\Gamma} &  & \ddot{\mathbf{s}} = \ddot{\boldsymbol{\theta}} \boldsymbol{\Gamma}
\end{align}

we can rewrite the \cref{eqn:mot_dyn} in the joints' space as:

\begin{equation}
    \label{eqn:mot_dyn_jointspace}
    \boldsymbol{\tau} = \boldsymbol{\Gamma} ^{-T} (\mathbf{I_m}\boldsymbol{\Gamma} ^{-1} \ddot{s} + \mathbf{K}_v \boldsymbol{\Gamma} ^{-1}\dot{s})
\end{equation}

Therefore, the \ac{EoM} of the multibody system can be rewritten as:

\begin{equation}
    \underbrace{\begin{bmatrix}
            \mathbf{M} _{\mathcal{B}}(q)     & \mathbf{M} _{\mathcal{B}S}(q)                                                      \\
            \mathbf{M} _{\mathcal{B}S} ^T(q) & \mathbf{M} _s(q) + \boldsymbol{\Gamma} ^{-T}\mathbf{I} _m\boldsymbol{\Gamma} ^{-1}
        \end{bmatrix}} _{\mathbf{\bar{M}}(q)}
    \begin{bmatrix}
        \dot{\mathrm{\mathbf{v}}} \\
        \ddot{\mathbf{s}}
    \end{bmatrix}+
    \underbrace{\begin{bmatrix}
            \mathbf{h} _{\mathcal{B}} \\
            \mathbf{h} _S
        \end{bmatrix}} _\mathbf{\bar{h}}(q,\boldsymbol{\nu}) =
    \underbrace{\begin{bmatrix}
            \mathbb{0} \\
            \boldsymbol{\Gamma} ^{-T}
        \end{bmatrix}} _{\mathbf{\bar{B}}}
    \boldsymbol{\tau} _m
    +
    \begin{bmatrix}
        \mathbf{J} _{\mathcal{B}} \\
        \mathbf{J} _S
    \end{bmatrix} ^T
    \begin{bmatrix}
        \mathbf{f} _{\mathcal{B}} \\
        \mathbf{f} _S
    \end{bmatrix}-
    \underbrace{\begin{bmatrix}
            \mathbb{0} \\
            \boldsymbol{\Gamma} ^{-T}\mathbf{K _v}\boldsymbol{\Gamma} ^{-1}
        \end{bmatrix}} _\mathbf{\bar{K _v}}
    \begin{bmatrix}
        \mathrm{\mathbf{v}} \\
        \dot{\mathbf{s}}
    \end{bmatrix}
\end{equation}

or, in a more compact form that will be used for computation as:

\begin{equation}
    \mathbf{\bar{M}}(q)\dot{\boldsymbol{\nu}} + \mathbf{\bar{h}}(q,\boldsymbol{\nu}) = \mathbf{\bar{B}}\boldsymbol{\tau} _m + \mathbf{J} ^T \mathbf{f} - \bar{\mathbf{K _v}}\boldsymbol{\nu}
\end{equation}

\subsection{Articulated Body Algorithm}

For the \textit{Lagrange-d'Alembert} formulation, in the case of systems with constant mass, the the virtual work $\delta W$ of the active and inertial forces through a virtual displacement $\delta \mathbf{q}$ is given by:

$$
    \delta W = \delta W _{act} + \delta W _{inert} = \sum _{i = 1} ^{N _B} \mathbf{f} _i ^T \delta \mathbf{q} + \sum _{i = 1} ^{N _B} \mathbf{p} ^A _i \delta \mathbf{v} _i
$$

where $\mathbf{f} _i$ is the generalized force and $\mathbf{p} ^A _i$ is the momentum of the $i$-th body. The momentum $\mathbf{p} ^A _i$ is given by:

$$
    \mathbf{p} ^A _i = \mathbf{v} _i \times ^* \mathbf{I} _i \mathbf{v} _i - \mathbf{X} ^* _0 f ^x _i
$$

where $\mathbf{v} _i$ is the velocity of the $i$-th body, $\mathbf{I} _i$ is the inertia matrix of the $i$-th body, and $\mathbf{X} _0$ is the spatial transformation from the world frame to the base frame. The generalized force $\mathbf{f} _i$ considering that the joints are massless is given by:

$$
    \mathbf{f} _i = \mathbf{S} ^T _i \mathbf{p} ^A _i
$$

where $\mathbf{S} _i$ is the motion subspace of the $i$-th link, which is a matrix with dimensions $6 \times N _{DOF}$ where $N _{DOF}$ is the number of joints. The Coriolis acceleration $\mathbf{c} _i$ is given by:

$$
    \mathbf{c} _i = \mathbf{v} _i \times ^* {} ^{ii}\mathbf{v} _J
$$

where $\mathbf{v} _i$ is the velocity of the $i$-th body and $\mathbf{v} _J$ is the velocity of the $i$-th joint. The velocity of the $ii$-th joint ${} ^{ii}\mathbf{v} _J$ is given by:

$$
    \mathbf{v} _i = \mathbf{X} _i \mathbf{v} _{\lambda _i} + {} ^{ii}\mathbf{v} _J
$$

where $\mathbf{X} _i$ is the spatial transformation from the $i$-th body to the parent body and $\mathbf{v} _{\lambda _i}$ is the velocity of the parent body. The velocity of the $ii$-th joint ${} ^{ii}\mathbf{v} _J$ is given by:

$$
    {} ^{ii}\mathbf{v} _J = \mathbf{S} _i \dot{\mathbf{q}} _i
$$

where $\mathbf{S} _i$ is the selection matrix of the $i$-th joint and $\dot{\mathbf{q}} _i$ is the velocity of the $i$-th joint.

If we consider the motor dynamics, the generalized force $\mathbf{f} _i$ is given by:

$$
    \mathbf{f} _i = \mathbf{S} ^T _i \mathbf{p} ^A _i + {} ^{ii} \mathbf{K} _v {} ^{ii}\mathbf{v} _J
$$

where $\mathbf{K} _v$ is the viscous friction matrix of the joints. This will act as a damping term in the joint dynamics.

Considering also the motor inertia $I _m$ the momentum $\mathbf{p} ^A _i$, for the aforementioned principle, will have the additional term $\mathbf{I} _m \mathbf{v} _J$:

$$
    \mathbf{p} ^A _i = \mathbf{v} _i \times ^* \mathbf{I} _i \mathbf{v} _i - \mathbf{X} ^* _0 f ^x _i + {} ^{ii}\mathbf{I} _m {} ^{ii}\mathbf{v} _J
$$


\begin{algorithm}[H]
    \caption{Articulated Body Algorithm}
    \label{alg:aba}
    \begin{algorithmic}[1]
        \REQUIRE Joints' velocities $\dot{\mathbf{q}}$, torques applied $\boldsymbol{\tau}$
        \FOR{$i = 1 \text{ to } N_B$}
        \STATE{$[\mathbf{X}_J, \mathbf{S}_i] = \text{jcalc}(\text{jtype}(i), \dot{\mathbf{q}}_i)$}
        \STATE $\mathrm{\mathbf{v}}_J = \mathbf{S}_i \dot{\mathbf{q}}_i$
        \STATE $^i\mathbf{X}_{\lambda(i)} = \mathbf{X}_J\mathbf{X}_T (i)$
        \IF{$\lambda_i = 0$}
        \STATE $\mathrm{\mathbf{v}}_i = \mathrm{\mathbf{v}}_J$
        \STATE $\mathbf{c}_i = \mathbf{0}$
        \ELSE
        \STATE $\mathrm{\mathbf{v}}_i = ^i\mathbf{X} _{\lambda(i)}\mathrm{\mathbf{v}}_{\lambda(i)} + \mathrm{\mathbf{v}}_J$
        \STATE $\mathbf{c}_i = \mathrm{\mathbf{v}}_i \times ^* \mathrm{\mathbf{v}}_J$
        \ENDIF
        \STATE $\mathbf{I}_i ^A = \mathbf{I}_i$
        \STATE $\mathbf{p}_i ^A = \mathrm{\mathbf{v}}_i \times^* \mathbf{I}_i\mathrm{\mathbf{v}}_i - ^i\mathbf{X} _0 ^* f ^* _i - \boldsymbol{\Gamma} ^{-T}\mathbf{K}_v \boldsymbol{\Gamma} ^{-1} \mathbf{\dot{q}}$
        \ENDFOR

        % Pass 2
        \FOR{$i = N_B \text{ to } 1$}
        \STATE $\mathbf{U}_i = \mathbf{I}_i \mathbf{S}_i$
        \STATE $\mathbf{D}_i = \mathbf{S}_i ^T \mathbf{U}_i$
        \STATE $\mathbf{u}_i = \boldsymbol{\Gamma}^{-T}\boldsymbol{\tau}_i - \mathbf{S}_i\mathbf{p}_i^A$
        \IF{$\lambda_i \neq 0$}
        \STATE $\mathbf{I}^a = \mathbf{I}^a_i - \mathbf{U}_i\mathbf{D}_i ^{-1}\mathbf{U}_i ^T$
        \STATE $\mathbf{p}^a = \mathbf{p}^a_i + \mathbf{I}^a\mathbf{C}_i + \mathbf{U}_i\mathbf{D}^{-1}\mathbf{u}_i$
        \STATE $\mathbf{I}^A_{\lambda(i)} = \mathbf{I}^A_{\lambda(i)} + {}^{\lambda(i)}\mathbf{X}_i^* \mathbf{I}^a {}^i\mathbf{X}_{\lambda(i)}$
        \STATE $\mathbf{p}^A_{\lambda(i)} = \mathbf{p}^A_{\lambda(i)} + {}^{\lambda(i)}\mathbf{X}_i^*\mathbf{p}^a$
        \ENDIF
        \ENDFOR

        % Pass 3
        \FOR{$i = 1 \text{ to } N_B$}
        \IF{$\lambda_i = 0$}
        \STATE $\mathbf{a}' = -\mathbf{a}_g$
        \ELSE
        \STATE $\mathbf{a}' = {}^{\lambda(i)}\mathbf{X}_i \mathbf{a}_{\lambda(i)} + \mathbf{c}_i$
        \STATE $\ddot{\mathbf{q}}_i = \mathbf{D}^{-1} (\mathbf{u}_i - \mathbf{U}_i^T\mathbf{a}')$
        \STATE $\mathbf{a}_i = \mathbf{a}' + \mathbf{S}_i\mathbf{\ddot{q}}_i$
        \ENDIF
        \ENDFOR
    \end{algorithmic}
\end{algorithm}
